\documentclass[14pt]{extarticle}

\usepackage{homeworktemplate}
\usepackage[askip=3mm, bskip=3mm]{mylisting}
\usepackage[stable]{footmisc}
\usepackage{csquotes}
\usetikzlibrary{positioning}
\usetikzlibrary{shapes.geometric}
\usetikzlibrary {arrows.meta}

\title{Домашняя работа 3 \\ The Guessing Game}

\begin{document}

\maketitle

\tableofcontents

\section{Задание} \label{requirements}

    В этом домашнем задании вам предлагается реализовать простейший случай
    игры Guessing Game.
    Её суть сводится к тому, что программа \enquote{загадывает} число в известном
    диапазоне, а пользователь должен это число угадать с нескольких попыток.

    \subsection{Требования}

        \begin{enumerate}

            \item Программа после начала выполнения должна \enquote{загадывать} случайное число в
                диапазоне от 0 до 9.
                Программа на протяжении всей дальнейшей работы не должна менять это число.

            \item Количество попыток пользователя не ограничено.
                Программа штатно завершает работу только после удачной попытки угадать число.

            \item Программа должна выводить строку \framebox{Guess a number from 0 to 9: }
                (обратите внимание на пробел в конце строки) перед каждой попыткой
                пользователя угадать число.

            \item Программа должна выводить \framebox{Wrong!}, когда введенное пользователем число
                не совпало с \enquote{загаданным}, и \framebox{Correct!}, когда совпало. 
            
        \end{enumerate}

    \subsection{Алгоритм}

        Алгоритм игры изображен на рис.~\ref{fig:algorithm}.

        \begin{figure}[h!]

            \centering

            \begin{tikzpicture}

                \tikzstyle{every node}=[draw]

                \node (start) [rounded corners=10pt] {Начало работы};
                \node (generation) [rectangle, below=1cm of start] {Генерация случайного числа};
                \node (input) [trapezium, trapezium right angle=120, trapezium left angle=60, below=1cm of generation] {Ввод числа пользователем};
                \node (decision) [diamond, aspect=2, below=1cm of input] {Числа равны?};
                \node (end) [rounded corners=10pt, below=1cm of decision] {Завершение работы};

                \node (no) [draw=none, fill=none, below right=0.1cm and 0.1cm of decision.east] {Нет};
                \node (no) [draw=none, fill=none, below right=0.1cm and 0.1cm of decision.south] {Да};
                 
                \draw[-{>[length=1mm]}] (start) -- (generation);
                \draw[-{>[length=1mm]}] (generation) -- (input);
                \draw[-{>[length=1mm]}] (input) -- (decision);
                \draw[-{>[length=1mm]}] (decision) -- (end);
                \draw[-{>[length=1mm]}] (decision.east) to[bend right=90] (input.east);

            \end{tikzpicture}

            \caption{Алгоритм игры}\label{fig:algorithm}

        \end{figure}

    \subsection{Ограничения}

        \begin{itemize}

            \item Можно считать, что в поток ввода всегда поступает строка,
                корректно преобразующаяся в число.
                Т.е., не нужно добавлять в программу валидацию ввода пользователя.
            
        \end{itemize}

\section{Порядок выполнения}

    \begin{enumerate}

        \item Создайте форк репозитория \url{https://github.com/czertyaka/CppDevCourse-hw-guessing-game}.

        \item Добавьте в форк файл \textit{main.cpp}.

        \item В файл \textit{main.cpp} добавьте код Guessing Game.

        \item Соберите проект и протестируйте решение.

        \item Создайте pull request из вашего форка в оригинальный репозиторий,
            в названии которого есть ваше ФИО.

    \end{enumerate}

\section{Критерии выполнения}

    \begin{itemize}

        \item Выполнены все требования из \ref{requirements}.

        \item Pull request проходит автоматизированные проверки.

        \item Код в pull request прошел ревью преподавателем.

    \end{itemize}

\section{Полезные ссылки}

    \begin{itemize}

        \item Операции ввода-вывода:

            \begin{itemize}

                \item \url{https://www.geeksforgeeks.org/cpp/basic-input-output-c/}

                \item \url{http://en.cppreference.com/w/cpp/io/cin.html}

                \item \url{http://en.cppreference.com/w/cpp/io/cout.html}

                \item \url{https://en.cppreference.com/w/cpp/io/manip.html}
                
            \end{itemize}

        \item Генерация случайный чисел:

            \begin{itemize}

                \item \url{http://en.cppreference.com/w/cpp/numeric/random.html}
                
            \end{itemize}
        
    \end{itemize}

\end{document}

