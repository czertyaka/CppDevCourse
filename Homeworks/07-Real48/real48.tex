\documentclass[14pt]{extarticle}

\usepackage{homeworktemplate}
\usepackage[askip=3mm, bskip=3mm]{terminal}
\usepackage[askip=3mm, bskip=3mm]{mylisting}
\usepackage{tcolorbox}
\usepackage{csquotes}

\title{Домашняя работа 6 \\ \enquote{Real48}}

\begin{document}

  \maketitle

  \tableofcontents

  \section{Описание задания}

    \subsection{Проблема}

    В Delphi существует тип данных \texttt{Real48}\footnotemark{}.
    Как можно понять из его наименования, он предназначен для представления чисел
    с плавающей точкой в 48-битах (т.е. в 6 байтах).

    \footnotetext{\url{https://docwiki.embarcadero.com/RADStudio/Athens/en/Internal\_Data\_Formats\_(Delphi)\#The\_Real48\_type}}

    Такой способ представления не входит в стандарт IEEE~754\footnote{\url{https://en.wikipedia.org/wiki/IEEE\_754}},
    который де-факто является умолчательным способом реализовывать
    арифметику чисел с плавающей точкой (как программно, так и аппаратно)
    в современных компьютерах.
    В системе типов Delphi он существует только для поддержания обратной
    совместимости\footnotemark{}, рекомендуется предпочитать ему
    типы \texttt{Double} и \texttt{Single}, которые удовлетворяют
    стандарту IEEE~754.

    \footnotetext{\url{https://docwiki.embarcadero.com/Libraries/Athens/en/System.Real48}}

    6 байт распределены на бит знака, экспоненту и мантиссу следующим образом
    (начиная с младшего бита):

    \begin{enumerate}
      \item экспонента ($e$) \textemdash \space 8 бит;
      \item мантисса ($f$) \textemdash \space 39 бит;
      \item знаковый бит ($s$) \textemdash \space 1 бит.
    \end{enumerate}

    Если условиться, что $e$, $f$ и $s$ представлены в виде беззнаковых
    целых чисел, то кодируемое в \texttt{Real48} число с плавающей точкой
    $F$ может быть вычислено по следующей формуле:

    \begin{equation}
      F = \left(-1\right)^s \cdot 2^{e - 129} \cdot \left(1 + \frac{f}{2^{39}}\right)
    \end{equation}

    Если принять мантиссу $f$ как простую последовательность бит, то это же
    уравнение можно написать в виде:

    \begin{equation}
      F = \left(-1\right)^s \cdot 2^{e - 129} \cdot 1.f
    \end{equation}

    Здесь запись $1.f$ \textemdash \space дробное число\footnotemark{} в двоичной системе
    счисления.

    \footnotetext{\url{https://en.wikipedia.org/wiki/Binary\_number\#Fractions}}

    У \texttt{Real48} всего одно специальной значение \textemdash \space
    ноль.
    Оно кодируется $e = 0$.
    Значения мантиссы и знакового бита не специфицируются и могут быть
    произовольными.

    Не предусмотрены специальные значения для представления NaN, бесконечности
    или ненормализированных чисел\footnote{\url{https://www.intel.com/content/www/us/en/docs/dpcpp-cpp-compiler/developer-guide-reference/2023-1/denormal-numbers.html}}.
    Последние при сохранении в \texttt{Real48} становятся нулем,
    NaN и бесконечность при попытке сохранить их в \texttt{Real48}
    приводят к ошибке.

    В домашнем задании вам предлагается написать программную реализацию \texttt{Real48}
    в виде C++ класса.
    Такой класс может быть полезен при сериализации/десериализации данных
    для отправки по сети приложению, написанному на Delphi и ожидающему
    данные в виде \texttt{Real48}.

    \subsection{Шаблон проекта}

    \subsection{Требования} \label{requirements}

  \section{Порядок выполнения}

    \begin{enumerate}

      \item Создайте форк репозитория \url{https://github.com/cppdevcourse/hw-real48}.

      \item Добавьте файл \textit{real48.cpp} и внесите изменения в \textit{real48.hpp}.

      \item Соберите проект и протестируйте решение (если получилось установить зависимости,
        необходимые для тестирования).
        Этот пункт не обязателен, потому что в pull request сборка и тесты будут выполнены
        в рамках GitHub Actions.

      \item Создайте pull request из вашего форка в оригинальный репозиторий,
        в названии которого есть ваше ФИО.

    \end{enumerate}

  \section{Критерии выполнения}

    \begin{itemize}

      \item Выполнены все требования из \ref{requirements}.

      \item Pull request проходит автоматизированные проверки.

      \item Код в pull request прошел ревью преподавателем.

    \end{itemize}

\end{document}

