\documentclass[14pt]{extarticle}

\usepackage{homeworktemplate}
\usepackage[askip=3mm, bskip=3mm]{mylisting}
\usepackage[stable]{footmisc}

\title{Домашняя работа 3 \\ Two Sum}

\begin{document}

\maketitle

\tableofcontents

\section{Задание}

    В этом домашнем задании вам предстоит решить частный случай проблемы Two Sum.
    Это задача нахождения двух чисел в массиве $\mathbf{nums}$:

    \begin{equation}
\mathbf{nums} = (nums_0, nums_1, \dots , nums_{N-2}, nums_{N-1})
    \end{equation}

    таких, что:

    \begin{equation}
nums_i + nums_j = t, \quad i,j \in [0, N)
    \end{equation}

    где $t$ --- некоторое заданное значение.

    Реализовать алгоритм поиска $i$ и $j$ нужно в функции \verb|two_sum|,
    которая объявлена следующим образом:

    \begin{myinplacelisting}[minted language=cpp]
#define ARRAY_SIZE 5

bool two_sum(
    const int nums[ARRAY_SIZE],
    const int target,
    std::size_t& index0,
    std::size_t& index1
);
    \end{myinplacelisting}

    Описание параметров функции:

    \begin{itemize}

        \item \verb|nums| --- массив $\mathbf{nums}$ из $N = 5$ целых чисел;

        \item \verb|target| --- значение $t$;

        \item \verb|index0| --- ссылка, в которую надо сохранить $i$;

        \item \verb|index1| --- ссылка, в которую надо сохранить $j$.

    \end{itemize}

    \subsection{Допущения для аргументов функции \texttt{two\_sum}}

        \begin{itemize}

            \item Сумма любых двух элементов массива \verb|nums| меньше \\
                \verb|std::numeric_limits<int>::max()|\footnotemark{}
                и больше \\ \verb|std::numeric_limits<int>::min()|.

            \item Объекты, на которые ссылаются \verb|index0| и \verb|index1|,
                инициализированы.

            \item В массиве \verb|nums| может быть более одной пары элементов,
                чья сумма равна \verb|target|.

        \end{itemize}

        \footnotetext{\url{https://en.cppreference.com/w/cpp/types/numeric_limits/max}}

    \subsection{Требования к функции \texttt{two\_sum}} \label{requirements}

        \begin{itemize}

            \item Функция должна возвращать \verb|true|, если нашлась хотя бы одна
                пара $(nums_i, nums_j)$, и \verb|false| --- в ином случае.

            \item Если нашлась хотя бы одна пара элементов, то после выполнения
                функции меньший из индексов любой пары должен быть записан в объект
                по ссылке \verb|index0| и больший по \verb|index1|.

            \item Функция не должна содержать неопределенного поведения, т.е.
                чтения/записи неинициализированной памяти, переполнения беззнаковых
                целых и др.

        \end{itemize}

\section{Порядок выполнения}

    \begin{enumerate}

        \item Создайте форк репозитория \url{https://github.com/czertyaka/CppDevCourse-hw3}.

        \item Добавьте в форк файл \textit{src/two-sum.cpp}.

        \item В файл \textit{src/two-sun.cpp} добавьте определение функции
            \verb|two_sum|.

        \item Соберите проект и протестируйте решение.

        \item Создайте pull request из вашего форка в оригинальный репозиторий,
            в название которого есть ваше ФИО.

    \end{enumerate}

\section{Критерии выполнения}

    \begin{itemize}

        \item Выполнены все требования из \ref{requirements}.

        \item Pull request проходит автоматизированные проверки.

        \item Код в pull request прошел ревью преподавателем.

    \end{itemize}

\section{Ассимптотическая эффективность\footnotemark{}}

    \footnotetext{Необязательный параграф}

    Наивный подход к решению проблемы Two Sum для несортированного массива
    дает ассимптотическую эффективность\footnotemark{} $O(n^2)$.
    Подумайте, как можно улучшить временную сложность алгоритма.

    \footnotetext{\url{https://en.wikipedia.org/wiki/Big\_O\_notation}}

\end{document}
