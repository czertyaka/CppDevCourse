 \documentclass[14pt]{extarticle}

\usepackage{projectstemplate}
\usepackage[askip=3mm, bskip=3mm]{terminal}
\usepackage[askip=3mm, bskip=3mm]{mylisting}

\title{Библиотека идентификации и аутентификации}

\begin{document}

\maketitle

\tableofcontents

\section{Термины и определения}

\begin{itemize}

 \item \textbf{Идентификация} --- поиск учетной записи пользователя в базе
  данных по переданным учетным данным. 
  Чаще всего для поиска используется логин или имя пользователя.

 \item \textbf{Аутентификация} --- подтверждение подлинности пользователя
  по переданным учетным данным.
  Чаще всего для этого проверяется пароль пользователя.

 \item \textbf{Хэш-сумма (хэш)} --- массив данных фиксированной длины, который
  получается из другого массива данных произвольной длины при помощи алгоритма
  хэширования.
  Любое изменение в исходном массиве данных влияет на хэш-сумму.

 \item \textbf{Криптографическая хэш-сумма} --- хэш-сумма, полученная в
  при помощи одного из криптографически стойких алгоритмов хэширования.
  Одним из свойств таких алгоритмов является необратимость преобразования:
  невозможно получить массив исходных данных, зная только его хэш.
  Такие хэш-суммы широко используются для хранения паролей в базах данных.
  В таких случаях, в базе данных вместо пароля хранится его хэш.
  При аутентификации пользователя вычисляется хэш полученного от него пароля
  и сравнивается с эталонным в базе данных.

  Одним из примеров криптографически стойких алгоритмов хэширования является
  SHA256\footnotemark{}.

  \footnotetext{\url{https://ru.wikipedia.org/wiki/SHA-2}}

 \item \textbf{Соль (криптография)} --- массив данных, который используется в
  алгоритме хэширования для вычисления хэша.
  Можно считать, что хэш-преобразование выполняется над некоторой комбинацией
  массива исходных данных и соли, например, над результатом их конкатенации
  (присоединения друг к другу).
  Существуют и более сложные механизмы комбинации соли и массива исходных
  данных.
  Один из них определяет стандарт PBKDF2\footnotemark{}.

  \footnotetext{\url{https://en.wikipedia.org/wiki/PBKDF2}}
  
  Необходима для усложнения криптоанализа злоумышленником в случае утечки
  базы данных.
  Часто пользователи используют не очень сложные пароли (например,
  \textit{12345678}), и у злоумышленника появляется возможность перебрать
  самые распространенные варианты, вычислить их хэш-суммы и сравнить их с хэшами
  в базе данных\footnotemark{}.
  Чем больше в базе данных учетных записей пользователей, тем выше шанс, что
  вычисленный хэш совпадет с каким-нибудь эталонным хэшем из базы.
  Если "солить" хэши, то злоумышленнику для взлома базы перебором необходимо
  будет так же знать и соль.

  \footnotetext{\url{https://en.wikipedia.org/wiki/Dictionary_attack}}

 \item \textbf{Динамическая соль} --- соль, которая генерируется для каждого
  входного массива данных индивидуально.
  При таком подходе для одного и того же пароля получаются разные хэш-суммы,
  что затрудняет криптоанализ базы данных.
  Криптоаналитику не удается установить факт использования одинакового пароля
  разными пользователями.
  При таком подходе допускается хранение соли вместе с хэш-суммами паролей
  в той же базе данных.

\end{itemize}

\section{Задание}

\section{Рекомендации по выполнению}

\end{document}
