\documentclass[compress, 8pt]{beamer}

\usepackage{presentationtemplate}
\usepackage[askip=3mm, bskip=3mm]{terminal}
\usepackage[linenosfontsize=\tiny, askip=3mm, bskip=3mm]{mylisting}
\usepackage{csquotes}

\newtcolorbox{task}{
    colback=yellow!50!white,
    boxrule=0.02cm,
    colframe=black,
    sharp corners,
    left=0mm,
    right=0mm,
    top=0mm,
    bottom=0mm,
    before upper={\textbf{Задание}:\:},
}

\title{Исключения}

\begin{document}

  \frame[plain]{\titlepage}

  \begin{frame}[fragile]

    \frametitle{Синтаксис}

    Ключевые слова:

    \begin{itemize}
      \item \texttt{throw}
      \item \texttt{try}
      \item \texttt{catch}
    \end{itemize}

  \end{frame}

  \begin{frame}[fragile]

    \frametitle{Синтаксис}

    Типовой упрощенный пример использования исключений:

    \begin{myinplacelisting}[minted language=cpp]
int foo(unsigned i) {
    const int arr[] = {1, 2, 3};
    if (i > 2) {
        throw std::out_of_range("Too large index");
    }
    return arr[i];
}

unsigned get_index();

int main() {
    const auto index = get_index();
    try {
        std::println("arr[{}] = {}", index, foo(index));
    }
    catch (const std::exception& err) {
        std::println(stderr, "Error: {}", err.what());
    }
}
    \end{myinplacelisting}

  \end{frame}

  \begin{frame}[fragile]

    \frametitle{Обработка ошибок}

    В C++ существует два подхода к обработке ошибок:

    \hfill \break

    \begin{itemize}
      \item Передача информации об ошибке в возвращаемом значении.
      \item Механизм исключений.
    \end{itemize}

    \hfill \break

    Каждый из них имеет свои преимущества и недостатки.
    Исключения считаются более \enquote{идиоматическим} способом.
    Рассмотрим сначала альтернативный способ.

  \end{frame}

  \begin{frame}[fragile]

    \frametitle{Ошибка в возвращаемом значении}

    \textbf{Вариант 1} \textemdash \space в возвращаемом значении код ошибки,
    а результаты выполнения передается по ссылке.

    \begin{myinplacelisting}[minted language=cpp]
enum class Result {
    OK,
    OUT_OF_RANGE
};

Result foo(unsigned i, int& result) {
    constexpr int arr[] = {1, 2, 3};
    if (i > 2) {
        return Result::OUT_OF_RANGE;
    }
    result = arr[i];
    return Result::OK;
}
    \end{myinplacelisting}

  \end{frame}

  \begin{frame}[fragile]

    \frametitle{Ошибка в возвращаемом значении}

    \textbf{Вариант 1} \textemdash \space в возвращаемом значении код ошибки,
    а результаты выполнения передается по ссылке.

    \begin{myinplacelisting}[minted language=cpp]
unsigned get_index();

int main() {
    int result {};
    switch (foo(get_index(), result)) {
        case Result::OK:
            std::println("Result: {}", result);
            break;
        case Result::OUT_OF_RANGE:
            std::println(stderr, "Error: out of bounds");
        default:
            std::println(stderr, "Error: unknown");
            break;
    }
}
    \end{myinplacelisting}

  \end{frame}

  \begin{frame}[fragile]

    \frametitle{Ошибка в возвращаемом значении}

    \begin{task}
      Подумайте, какие преимущества и недостатки у \textbf{варианта 1}.
    \end{task}

  \end{frame}

  \begin{frame}[fragile]

    \frametitle{Ошибка в возвращаемом значении}

    \textbf{Вариант 2} \textemdash \space информация об ошибке и результат
    операции объединены в один тип.

    \begin{myinplacelisting}[minted language=cpp]
struct Result {
    enum class Error {
        NONE,
        OUT_OF_RANGE
    };
    Error error = Error::NONE;
    int value;
};

Result foo(unsigned i) {
    constexpr int arr[] = {1, 2, 3};
    if (i > 2) {
        return {Result::Error::OUT_OF_RANGE};
    }
    Result result {};
    result.value = arr[i];
    return result;
}
    \end{myinplacelisting}

  \end{frame}

  \begin{frame}[fragile]

    \frametitle{Ошибка в возвращаемом значении}

    \textbf{Вариант 2} \textemdash \space информация об ошибке и результат
    операции объединены в один тип.

    \begin{myinplacelisting}[minted language=cpp]
unsigned get_index();

int main() {
    const auto result = foo(get_index());
    switch (result.error) {
        case Result::Error::NONE:
            std::println("Result: {}", result.value);
            break;
        case Result::Error::OUT_OF_RANGE:
            std::println(stderr,
                "Error: out of bounds");
        default:
            break;
    }
}
    \end{myinplacelisting}

  \end{frame}

  \begin{frame}[fragile]

    \frametitle{Ошибка в возвращаемом значении}

    \begin{task}
      Подумайте, какие преимущества и недостатки у \textbf{варианта 2}.
    \end{task}

  \end{frame}

  \begin{frame}[fragile]

    \frametitle{Unrepresentable state}

    Передача ошибки через возвращаемое значение \textemdash \space
    не универсальное решение.
    Возвращаемых значений нет у \textit{конструкторов классов}.

    \begin{myinplacelisting}[minted language=cpp]
class Month {
public:
    Month(const unsigned number) {
        if (number >= 12) {
            // how do we handle this?
        }
    }
    // ...
};
    \end{myinplacelisting}

  \end{frame}

  \begin{frame}[fragile]

    \frametitle{Unrepresentable state}

    Одним из способов решения проблемы \textemdash \space
    создание методов, указывающих на консистентность
    состояния объекта.

    \begin{myinplacelisting}[minted language=cpp]
class Month {
    std::uint8_t number_;
public:
    Month(std::uint8_t number) :
        number_(number) {}
    bool Valid() const {
        return number_ < 12;
    }
};

void foo(const Month& month) {
    if (month.Valid()) {
        // do stuff ...
    }
}
    \end{myinplacelisting}

  \end{frame}

  \begin{frame}[fragile]

    \frametitle{Unrepresentable state}

    Как правило, лучше вообще не давать создавать объект с
    неконсистентным состояним, делая его таким образом
    \enquote{непредставимым} (\textit{unrepresentable state}).

    \begin{myinplacelisting}[minted language=cpp]
class Month {
    std::uint8_t number_;
public:
    Month(std::uint8_t number) : number_(number) {
        if (number_ > 11) {
            throw std::out_of_range(
                "There's only 12 months"
            );
        }
    }
};
    \end{myinplacelisting}

  \end{frame}

  \begin{frame}[fragile]

    \frametitle{Выброс исключения}

    Исключения выбрасываются ключевым словом \texttt{throw}.
    Исключение может быть объектом любого типа.

    \begin{myinplacelisting}[minted language=cpp]
throw 1;
throw "hello!";
throw (1.0f * 2);
throw new int;
throw Foo{};
throw std::runtime_error("hello!");
    \end{myinplacelisting}

  \end{frame}

  \begin{frame}[fragile]

    \frametitle{Отлов исключений}

    Для отлова исключений используется блок \texttt{try} \slash \texttt{catch}.
    Семантика отлова исключений позволяет фильтровать их по \textit{типу}.

    \begin{myinplacelisting}[minted language=cpp]
void foo(int i) {
    switch (i) {
        case 1:  throw int{};
        case 2:  throw double{};
        default: throw nullptr;
    }
}

void bar(int i) {
    try {
        foo(i);
    }
    catch (const double err) {
        // `throw double` will be handled here
    }
    // all other `throw`s are not handled at all
}
    \end{myinplacelisting}

  \end{frame}

  \begin{frame}[fragile]

    \frametitle{Отлов исключений}

    В одном \texttt{try} \slash \texttt{catch} блоке можно отловить более
    одного типа исключений.
    Последовательность блоков \texttt{catch} важна:
    при возникновении неоднозначности в выборе блока будет выбран первый.

    \begin{myinplacelisting}[minted language=cpp]
void bar(int i) {
    try {
        foo(i);
    }
    catch (const double err) {
        std::println("Error: {}", err);
    }
    catch (const int err) {
        std::println("Error: {}", err);
    }
}
    \end{myinplacelisting}

  \end{frame}

  \begin{frame}[fragile]

    \frametitle{Отлов исключений}

    Для отлова исключения любого типа используется конструкция
    \texttt{catch(...)}.

    \begin{myinplacelisting}[minted language=cpp]
void bar(int i) {
    try {
        foo(i);
    }
    catch (const double err) {
        std::println("Error: {}", err);
    }
    catch (...) {
        // will handle all unhandled exceptions
        // but we lack error object here
        std::println("Error: unknown");
    }
}
    \end{myinplacelisting}

  \end{frame}

  \begin{frame}[fragile]

    \frametitle{Stack unwinding}

  \end{frame}

  \begin{frame}[fragile]

    \frametitle{\texttt{noexcept} функции}

  \end{frame}

  \begin{frame}[fragile]

    \frametitle{Исключения и деструкторы}

  \end{frame}

  \begin{frame}[fragile]

    \frametitle{Исключения из стандартной библиотеки}

  \end{frame}

  \begin{frame}[fragile]

    \frametitle{Собственные типы исключений}

  \end{frame}

\end{document}
