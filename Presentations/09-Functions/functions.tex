% !TeX TS-program = xelatex

\documentclass[compress, 8pt]{beamer}

\usepackage{presentationtemplate}
\usepackage[askip=3mm, bskip=3mm]{terminal}
\usepackage[linenosfontsize=\tiny, askip=3mm, bskip=3mm]{mylisting}
\usepackage{tcolorbox}
\usepackage{tikz}
\usetikzlibrary{positioning}
\usepackage{csquotes}

\newtcolorbox{task}{
    colback=yellow!50!white,
    boxrule=0.02cm,
    colframe=black,
    sharp corners,
    left=0mm,
    right=0mm,
    top=0mm,
    bottom=0mm,
    before upper={\textbf{Задание}:\:},
}

\title{Функции}

\begin{document}

\frame[plain]{\titlepage}

\begin{frame}[fragile]

    \frametitle{Функции}

    \textbf{Функции} в C++ связывают последовательность инструкций (тело функции)
    с именем (идентификатором) и набором параметров (возможно пустым).

    \hfill\break
    Одна из основных целей функций --- разбивать сложные вычисления на осмысленные
    части и давать им имена.

    \hfill\break
    После вызова функции:

    \begin{itemize}
        \item инициализируются параметры из аргументов;
        \item выполняются инструкции тела функции.
    \end{itemize}

\end{frame}

\begin{frame}[fragile]

    \frametitle{Объявление функции}

    В объявление функции входят:

    \begin{itemize}
        \item имя функции;
        \item параметры функции (без имен);
        \item спецификаторы \verb|constexpr|, \verb|inline|, \verb|static| и др.
    \end{itemize}

    \hfill\break
    Ограничения на количество объявлений одной и той же функции нет.
    Несколько эквивалентных объявлений:

    \centering
    \hfill\break
    \begin{tikzpicture}[node distance=3cm]

        \node (declarations) {
            \begin{myinplacelisting}[minted language=cpp, width=6cm]
int foo(double d, char c);
int foo(const double d, char c);
int foo(double, char);
void foo(double d, char c);
            \end{myinplacelisting}
        };

        \node (symbol) [right=of declarations] { \verb|_Z3foodc| };

        \draw[->, thick] (declarations.east) -- (symbol.west)
            node[midway, above] {Декорирование};

    \end{tikzpicture}

    \begin{task}
        Объясните, зачем нужны объявление и декорирование функции.
    \end{task}

\end{frame}

\begin{frame}[fragile]

    \frametitle{Объявление функции}

    Функция должна быть объявлена \textit{до} вызова.

    \myinputlisting[minted language=cpp]
        {Presentations/09-Functions/declaration-misuse/}
        {main.cpp}

    \begin{terminalwindow}
!\shellcommand{g++ main.cpp}!
main.cpp: In function ‘int !\textcolor{teal}{main}!()’:
main.cpp:2:5: !\textcolor{red}{error}!: ‘foo’ was not declared in this scope
    2 |     !\textcolor{red}{foo}!(); // compilation error
      |     !\textcolor{red}{\^{}~~}!
    \end{terminalwindow}

\end{frame}

\begin{frame}[fragile]

    \frametitle{Определение функции}

    Функция может быть определена как отдельно от объявления, так и вместе с ним.

    \begin{myinplacelisting}[minted language=cpp]
void foo(); // declaration

void bar() { // declaration and definition
    std::cout << "bar" << std::endl;
}

int main() {
    foo();
    bar();
}

void foo() { // definition
    std::cout << "foo" << std::endl;
}
    \end{myinplacelisting}

\end{frame}

\begin{frame}[fragile]

    \frametitle{Определение функции}

    \hfill\break
    Функция должна быть определена в программе только один раз\footnotemark{}.

    \footnotetext{Несколько определений являются нарушением
        \href{https://en.cppreference.com/w/cpp/language/definition}{One Defition Rule}
        (ODR violation).}

    \begin{task}
        Объясните, почему не допускается определять функцию в заголовочном файле,
        если он используется в нескольких единицах трансляции, и на каком этапе трансляции
        происходит ошибка.
    \end{task}

    \begin{columns}[T]

        \begin{column}{0.5\textwidth}

            \myinputlisting[minted language=cpp]
                {Presentations/09-Functions/ODR-violation/}
                {foo.hpp}

            \myinputlisting[minted language=cpp]
                {Presentations/09-Functions/ODR-violation/}
                {foo.cpp}

            \myinputlisting[minted language=cpp]
                {Presentations/09-Functions/ODR-violation/}
                {main.cpp}

        \end{column}

        \begin{column}{0.5\textwidth}

            \begin{terminalwindow}
!\shellcommand{g++ main.cpp foo.cpp}!
/usr/bin/ld: /tmp/cctdCoR8.o: in function `foo()':
foo.cpp:(.text+0x0): multiple definition of `foo()'; /tmp/cchYkLGd.o:main.cpp:(.text+0x0): first defined here
collect2: error: ld returned 1 exit status
            \end{terminalwindow}

        \end{column}

    \end{columns}

\end{frame}

\begin{frame}[fragile]

    \frametitle{Параметры и аргументы}

    \begin{itemize}

        \item \textbf{Параметры} --- объекты, с которыми функция \enquote{работает} в своем теле.

        \item \textbf{Аргументы} --- объекты, которые вызывающий контекст передает функции
            для инициализации переменных.

    \end{itemize}

    \hfill\break
    \begin{task}
        Назовите параметры и аргументы в коде на листинге ниже.
    \end{task}

    \begin{myinplacelisting}[minted language=cpp]
double bar();
void foo(int i, double d);

int main() {
    foo(1 * 2, bar());
}
    \end{myinplacelisting}

\end{frame}

\begin{frame}[fragile]

    \frametitle{Аргументы по-умолчанию}

    \hfill\break
    В объявлении функции некоторым параметрам можно задать аргументы по-умолчанию.
    Аргументы по-умолчанию может иметь непрерывный набор последних параметров.
    Не допускается повторно указывать аргументы по-умолчанию в определении функции.

    \begin{myinplacelisting}[minted language=cpp]
#include <iostream>

void foo(int x, int y = 0); // y has default argument
void bar(int x = 0, int y); // compilation error
void bar(int x = 0, int y, int z = 0); // compilation error

void foo(int x, int y = 0) {} // compilation error

void foo(int x, int y/* = 0*/) {
    std::cout << "x=" << x << ", y=" << y << std::endl;
}

int main() {
    foo(1); // is equal with foo(1, 0);
    foo(1, 2); // override default argument
}
    \end{myinplacelisting}

\end{frame}

\begin{frame}[fragile]

    \frametitle{Передача аргумента по значению}

    Значения аргументов \textit{всегда}\footnotemark{} копируются в функцию:

    \footnotetext{За исключением оптимизации
        \href{https://en.cppreference.com/w/cpp/language/copy\_elision}{Copy elision}.}

    \begin{columns}[T]

        \begin{column}{0.5\textwidth}

            \myinputlisting[minted language=cpp]
                {Presentations/09-Functions/pass-by-value/}
                {main.cpp}

        \end{column}

        \begin{column}{0.5\textwidth}

            \begin{terminalwindow}
!\shellcommand{g++ main.cpp -o main}!
!\shellcommand{./main}!
&x=0x7ffc91fee7ac
&x=0x7ffc91fee78c
x=1
            \end{terminalwindow}

            Такая передача аргумента называется \textbf{передачей по значению}
            (pass by value).

        \end{column}

    \end{columns}

\end{frame}

\begin{frame}[fragile]

    \frametitle{Передача аргумента по ссылке}

    \hfill\break
    Если необходимо менять переданный в функцию аргумент, можно передать ссылку или
    указатель (pass by reference).

    \begin{columns}[T]

        \begin{column}{0.5\textwidth}

            \myinputlisting[minted language=cpp]
                {Presentations/09-Functions/pass-by-ref/}
                {main.cpp}

        \end{column}

        \begin{column}{0.5\textwidth}

            \begin{terminalwindow}
!\shellcommand{g++ main.cpp -o main}!
!\shellcommand{./main}!
&x=0x7ffc91fee7ac
&p=0x7ffcab72b540
&p=0x7ffcab72b528
x=-1
            \end{terminalwindow}

            Объект ссылочного типа (ссылка или указатель) все равно будут скопирован,
            но позволит изменить объект, на который он ссылается.

        \end{column}

    \end{columns}

\end{frame}

\begin{frame}[fragile]

    \frametitle{Аргументы функции: order of evaluation\footnote{\url{https://en.cppreference.com/w/cpp/language/eval\_order.html}}}

    \begin{columns}[T]

        \begin{column}{0.5\textwidth}

            \begin{myinplacelisting}[minted language=cpp]
#include <iostream>

int foo() {
    std::cout << "foo\n";
    return 1;
}

int bar() {
    std::cout << "bar\n";
    return 2;
}

void sum(int a, int b) {
    std::cout << a + b
        << '\n';
}

int main() {
    sum(foo(), bar());
}
            \end{myinplacelisting}

        \end{column}

        \begin{column}{0.5\textwidth}

            \begin{terminalwindow}
!\shellcommand{g++ -O0 main.cpp}!
!\shellcommand{./a.out}!
bar
foo
3
!\shellcommand{clang++ -O0 main.cpp}!
!\shellcommand{./a.out}!
foo
bar
3
            \end{terminalwindow}

            Порядок вычисления значений аргументов функции является
            \textbf{unspecified behaviour}\footnotemark{}.

            \footnotetext{\url{https://en.wikipedia.org/wiki/Unspecified\_behavior}}

        \end{column}

    \end{columns}
    
\end{frame}

\begin{frame}[fragile]

    \frametitle{Возвращаемое значение функции}

    Функция может возвращать результат своего выполнения в \textbf{возвращаемоем значении}
    (return value).
    Возвращаемое есть в определении функции, но не является частью её \textbf{сигнатуры}
    (function signature\footnotemark{}).

    \footnotetext{\url{https://en.cppreference.com/w/cpp/language/function}}

    \begin{myinplacelisting}[minted language=cpp]
int foo(); // foo returns value of type int
auto foo() -> int; // same, C++11
    \end{myinplacelisting}

    Отсутствие возвращаемого значения выражается с помощью типа \verb|void|.

    \begin{myinplacelisting}[minted language=cpp]
void foo(); // foo has no return value
    \end{myinplacelisting}

\end{frame}

\begin{frame}[fragile]

    \frametitle{Возвращаемое значение функции}

    Вызов в функций в выражениях создает временный объект с возвращаемым
    значением функции.

    \begin{myinplacelisting}[minted language=cpp]
int foo();
void bar();

int main() {
    int x = foo();
    int* p = foo(); // compilation error
    int y = bar(); // compilation error
}
    \end{myinplacelisting}

\end{frame}

\begin{frame}[fragile]

    \frametitle{Инструкция \texttt{return}}

    \begin{columns}[T]

        \begin{column}{0.5\textwidth}

            Если у функции нет возвращаемого значения, то инструкция \verb|return|
            просто прекращает ее выполнение:

            \begin{myinplacelisting}[minted language=cpp]
#include <iostream>

void foo() {
    std::cout << __LINE__
        << std::endl;
    return;

    // dead code
    std::cout << __LINE__
        << std::endl;
}
            \end{myinplacelisting}

        \end{column}

        \begin{column}{0.5\textwidth}

            Если у функции есть возвращаемое значение, то оно указывается
            совместно с инструкцией \verb|return|.

            \begin{myinplacelisting}[minted language=cpp]
bool condition();

int foo() {
    if (condition()) {
        return 1;
    }
    return 0;
}
            \end{myinplacelisting}

        \end{column}

    \end{columns}

\end{frame}

\begin{frame}[fragile]

    \frametitle{NRVO}

    Возвращаемое значение обычно копируется в вызывающий контекст.
    Иногда компилятор может применить \textit{необязательную} варицию оптимизации
    Copy elision, известную как Named Return Value Optimization (NRVO).
    Эта оптимизация помещает возвращаемое значение сразу в вызывающий контекст
    \textit{без копирования}.
    Эта оптимизация может дать прирост производительности, если возвращаемое значение
    --- большой объект составного типа.

    \begin{myinplacelisting}[minted language=cpp]
struct MyLargeObject { /* ... */ };

MyLargeObject foo() {
    MyLargeObject obj {};
    return obl; // may be optimized with copy elision
}
    \end{myinplacelisting}

\end{frame}

\begin{frame}[fragile]

    \frametitle{Рекурсия}

    Вызов функции из тела той же функции называется \textbf{рекурсией} (recursion).

    \begin{myinplacelisting}[minted language=cpp]
unsigned factorial(unsigned n) {
    if (n < 2) {
        return 1;
    }
    return n * factorial(n - 1);
}
    \end{myinplacelisting}

\end{frame}

\begin{frame}[fragile]

    \frametitle{Переполнение стека}

    Неконтролируемая (бесконечнaя) рекурсия или рекурсия с очень большим число рекурсивных
    вызовов может привести к ошибке \textbf{переполнения стека}\footnotemark{}
    (stack overflow), потому что для каждого вызова функции на стеке аллоцируется память.

    \footnotetext{\url{https://en.wikipedia.org/wiki/Stack\_Overflow}}

    \begin{task}
        Найдите возможное переполнение стека в программе на листинге ниже и предложите
        исправление.
    \end{task}

    \begin{myinplacelisting}[%
        minted language=cpp,
        minted options={
            fontsize=\normalsize,
            breaklines,
            breakanywhere,
            linenos,
            escapeinside=??,
            numbersep=2pt
        },
    ]
int factorial(int n) {
    if (n == 0 || n == 1) {
        return 1;
    }
    return n * factorial(n - 1);
}
    \end{myinplacelisting}

\end{frame}

\begin{frame}[fragile]

    \frametitle{Переполнение стека}

    Когда несколько функций имеют одно и то же имя, говорят, что они
    \textbf{перегружены}\footnotemark{} (overloaded).
    При вызове перегруженной функции компилятору на основе аргументов
    необходимо решить, какую из функций вызвать.
    Возможны ситуации, когда компилятор не в состоянии разрешить перегрузку.

    \footnotetext{\url{https://en.cppreference.com/w/cpp/language/overload\_resolution}}

    \begin{myinplacelisting}[minted language=cpp]
void foo(double);
void foo(int);

struct X {};

int main() {
    const X x {};
    const double y {};
    const int z {};
    foo(x); // compilation error
    foo(y);
    foo(z);
}
    \end{myinplacelisting}

\end{frame}

\begin{frame}[fragile]

    \frametitle{Связывание функций}

    Функции могут иметь внешнее и внутреннее связывание (external and internal linkage).
    Функции с внешним связыванием можно вызывать из других единиц трансляции
    (если доступно их объявление).
    Функции с внутренним связыванием доступны только из той единицы трансляции,
    в которой находится их опрделение.

    \hfill\break
    Есть два способа сделать связывание функции внутренним:

    \begin{itemize}

        \item ключевое слово \verb|static|:
            \begin{myinplacelisting}[minted language=cpp]
static void foo();
            \end{myinplacelisting}

        \item безымянное пространство имен:
            \begin{myinplacelisting}[minted language=cpp]
namespace {
    void foo();
}
            \end{myinplacelisting}

    \end{itemize}

\end{frame}

\begin{frame}[fragile]

    \frametitle{Связывание функций}

    \hfill\break
    \myinputlisting[minted language=cpp]
        {Presentations/09-Functions/linkage/}
        {foo.cpp}

    \myinputlisting[minted language=cpp]
        {Presentations/09-Functions/linkage/}
        {main.cpp}

\end{frame}

\begin{frame}[fragile]

    \frametitle{Встраиваемые функции}

    Функции, объявленные со спецификатором \verb|inline|\footnotemark{}, называются
    \textbf{встраиваемыми} (inline functions).
    Встраиваемые функции имеют следующие свойства:

    \footnotetext{\url{https://en.cppreference.com/w/cpp/language/inline}}

    \begin{itemize}
        \item определение встраиваемой функции должно быть в той же единице
            трансляции, где она вызывается;
        \item встраиваемые функции с внешним связыванием могут иметь более
            одного определения во всей программе и должны быть объявлены
            с \verb|inline| в каждой единице трансляции.
    \end{itemize}

    Компилятор может, но не обязан, применить встраиваение\footnotetext{} такой
    функции в контекст вызова.
    Не рекомендуется делать встраиваемыми очень большие функции, поскольку это
    может привести к увеличение размера бинарного файла.

    \footnotetext{\url{https://en.wikipedia.org/wiki/Inline\_expansion}}

\end{frame}

\begin{frame}[fragile]

    \frametitle{Встраиваемые функции}

    Пример использования \verb|inline| функции.

    \myinputlisting[minted language=cpp]
        {Presentations/09-Functions/inline/}
        {foo.hpp}

    \myinputlisting[minted language=cpp]
        {Presentations/09-Functions/inline/}
        {foo.cpp}

    \myinputlisting[minted language=cpp]
        {Presentations/09-Functions/inline/}
        {main.cpp}

\end{frame}

\begin{frame}[fragile]

    \frametitle{\texttt{constexpr} функции}

    C++ поддерживает два типа иммутабельности:

    \begin{itemize}
        \item \verb|const|;
        \item \verb|constexpr|\footnotemark{} --- для выражений, вычислить которые
            желательно во время компиляции.
    \end{itemize}

    \footnotetext{\url{https://en.cppreference.com/w/cpp/language/constant\_expression\#Core\_constant\_expressions}}

    \hfill\break
    Обычные функции нельзя использовать в \verb|constexpr|, поскольку
    их возвращаемое значение вычисляется во время выполнения программы, а не во время
    компиляции:

    \begin{myinplacelisting}[minted language=cpp]
int foo() { return 1; }

int main() {
    constexpr int x = foo(); // compilation error
}
    \end{myinplacelisting}

\end{frame}

\begin{frame}[fragile]

    \frametitle{\texttt{constexpr} функции}

    Пример \verb|constexpr| функции:

    \begin{myinplacelisting}[minted language=cpp]
unsigned get_from_user();

constexpr unsigned factorial(const unsigned n) {
    if (n < 2) {
        return 1;
    }
    return n * factorial(n - 1);
}

int main() {
    // must be evaluated at compile time
    constexpr unsigned f1 = factorial(9);

    // compilation error
    constexpr unsigned f2 = factorial(get_from_user());
}
    \end{myinplacelisting}

\end{frame}

\begin{frame}[fragile]

    \frametitle{Указатель на функцию}

    С функциями можно работать через указатели на них.

    \begin{myinplacelisting}[minted language=cpp]
void foo() {}
int bar(int x, int y) { return 1; }

void callback(int (*p)(int, int)) {
    const int x = (*p)(1, 2); // calling function which p
                              // points to
}

int main() {
    void (*pfoo)() = &foo; // pointer to function foo
    int (*pbar)(int, int) = &bar; // pointer to function bar
    callback(pbar);
}
    \end{myinplacelisting}

\end{frame}

\end{document}
