% !TeX TS-program = xelatex

\documentclass[compress]{beamer}

\usepackage{presentationtemplate}
\usepackage[fontsize=\scriptsize, askip=3mm, bskip=3mm]{terminal}
\usepackage[fontsize=\scriptsize, linenosfontsize=\tiny, askip=3mm, bskip=3mm]{mylisting}
\usepackage[dvipsnames, table]{xcolor}
\usepackage{tikz}
\usetikzlibrary{positioning}

\newenvironment{sixteenbit}{%
    \scriptsize
        \begin{center}
            \begin{tabular}{ |c|c|c|c|c|c|c|c|c|c|c|c|c|c|c|c|  }
                \hline
}{
            \hline
            \multicolumn{1}{c}{\color{gray}\tiny{15}} &
            \multicolumn{1}{c}{\color{gray}\tiny{14}} &
            \multicolumn{1}{c}{\color{gray}\tiny{13}} &
            \multicolumn{1}{c}{\color{gray}\tiny{12}} &
            \multicolumn{1}{c}{\color{gray}\tiny{11}} &
            \multicolumn{1}{c}{\color{gray}\tiny{10}} &
            \multicolumn{1}{c}{\color{gray}\tiny{9}} &
            \multicolumn{1}{c}{\color{gray}\tiny{8}} &
            \multicolumn{1}{c}{\color{gray}\tiny{7}} &
            \multicolumn{1}{c}{\color{gray}\tiny{6}} &
            \multicolumn{1}{c}{\color{gray}\tiny{5}} &
            \multicolumn{1}{c}{\color{gray}\tiny{4}} &
            \multicolumn{1}{c}{\color{gray}\tiny{3}} &
            \multicolumn{1}{c}{\color{gray}\tiny{2}} &
            \multicolumn{1}{c}{\color{gray}\tiny{1}} &
            \multicolumn{1}{c}{\color{gray}\tiny{0}} \\
            \end{tabular}
        \end{center}
}

\title{Фундаментальные типы данных}

\begin{document}

\frame[plain]{\titlepage}

\begin{frame}

    \frametitle{Что такое тип объекта?}

    Тип объекта
    \footnote{Помимо объектов типами также обладают
    ссылки, функции и выражения.}
    определяет:

    \begin{itemize}

        \item семантическое значение объекта, представленного
            набором бит в памяти;

        \item набор и реализацию операций, которые можно производить
            над объектом.

    \end{itemize}

\end{frame}

\begin{frame}[fragile]

    \frametitle{Семантика объекта}

    \myinputlisting[minted language=cpp]
        {Presentations/3-Fundamental-types/}
        {semantics.cpp}

    \begin{terminalwindow}
!\shellcommand{g++ semantics.cpp}!
!\shellcommand{./a.out}!
b=1, s=65, c=A
    \end{terminalwindow}

    \scriptsize

    Объекты, представленные одним и тем же набором бит в памяти, можно
    наделить разной семантикой: они будут по-разному интерпретироваться
    программистом, компилятором, средой выполнения и т.д.

\end{frame}

\begin{frame}[fragile]

    \frametitle{Набор операций над типом}

    \begin{columns}[T]

        \begin{column}{0.5\textwidth}

            \begin{myinplacelisting}[minted language=cpp]
int a = 0;
a + 1;
a * 2;

void foo() {}
// ...
foo();

float f = 0.1;
float* ptr = &f;
*ptr;
            \end{myinplacelisting}

            \scriptsize

            Для арифметических типов разрешены операции сложения
            и умножения, для функций разрешен оператор вызова,
            для указателей --- оператор разыменования и т.д.

        \end{column}

        \begin{column}{0.5\textwidth}

            \scriptsize

            Нельзя вызвать \verb|char| как функцию или разыменовать, как
            указатель, создать переменную с типом \verb|void| или
            применить оператор битового сдвига к переменной с типом
            плавающей точки.

            \begin{myinplacelisting}[minted language=cpp]
char c = 0;
c(); // compilation error
*c;  // compilation error

void v; // compilation error

float f;
f << 10; // compilation
         // error
            \end{myinplacelisting}

        \end{column}

    \end{columns}

\end{frame}

\begin{frame}[fragile]

    \frametitle{Набор операций над типом}

    \myinputlisting[minted language=cpp]
        {Presentations/3-Fundamental-types/}
        {indirection-on-int.cpp}

    \begin{terminalwindow}
!\shellcommand{g++ indirection-on-int.cpp}!
indirection-on-int.cpp: In function ‘int !\textcolor{teal}{main}!()’:
indirection-on-int.cpp:5:18: !\color{red}{error}!: invalid type argument of unary ‘*’ (have ‘int’)
    5 |     std::cout << !\textcolor{red}{*i}! << std::cout;
      |
    \end{terminalwindow}

    \scriptsize

    Допустимость проведения операции над объектом проверяется во время
    компиляции. Это прямое следствие того, что C++ --- статически
    типизированный язык.

\end{frame}

\begin{frame}[fragile]

    \frametitle{Различие в реализациях операций \\ над разными типами}

    \myinputlisting[minted language=cpp]
        {Presentations/3-Fundamental-types/}
        {sum.cpp}

    \begin{terminalwindow}
!\shellcommand{g++ -std=c++23 sum.cpp}!
    \end{terminalwindow}

    \scriptsize

    Выражения сумм \verb|u + u| и \verb|f + f| различаются семантикой, но не
    синтаксисом.

\end{frame}

\begin{frame}[fragile]

    \frametitle{Различие в реализациях операций \\ над разными типами}

    \scriptsize
    \verb|f = 2.0, u = 16384|

    \begin{sixteenbit}
        0 & \cellcolor{gray}{1} & 0 & 0 & 0 & 0 & 0 & 0 & 0 & 0 & 0 & 0 & 0 & 0 & 0 & 0 \\
    \end{sixteenbit}

    \scriptsize
    \verb|f + f = 4.0|

    \begin{sixteenbit}
        0 & \cellcolor{gray}{1} & 0 & 0 & 0 & \cellcolor{gray}{1} & 0 & 0 & 0 & 0 & 0 & 0 & 0 & 0 & 0 & 0 \\
    \end{sixteenbit}

    \scriptsize
    \verb|u + u = 32768|

    \begin{sixteenbit}
        \cellcolor{gray}{1} & 0 & 0 & 0 & 0 & 0 & 0 & 0 & 0 & 0 & 0 & 0 & 0 & 0 & 0 & 0 \\
    \end{sixteenbit}

\end{frame}

\begin{frame}[fragile]

    \frametitle{Классификация \\ фундаментальных типов}

    \scriptsize

    \begin{tikzpicture}
        \tikzstyle{every node}=[draw]
        %
        \node (1)                                       {Фундаментальные типы};
        \node (2)   [right=4.0cm of 1]                  {\verb|void|};
        \node (3)   [below right=0.1cm and 4.0cm of 1]  {\verb|std::nullptr_t|};
        \node (4)   [below=0.4cm of 1]                  {Арифметические типы};
        \node (5)   [below right=0.1cm and 1.5cm of 4]  {С плавающей точкой};
        \node (6)   [right=0.8cm of 5]                  {\verb|float|};
        \node (7)   [below=0.2cm of 6]                  {\verb|double|};
        \node (8)   [below left=0.2cm and -1.2cm of 7]  {\verb|long double|};
        \node (9)   [below=0.4cm of 4]                  {Целочисленные (интегральные)};
        \node (10)  [below right=0.1cm and 1.5cm of 9]  {\verb|bool|};
        \node (11)  [below right=0.8cm and 1.5cm of 9]  {Символьные};
        \node (12)  [below right=0.8cm and 1.5cm of 11] {\verb|char|};
        \node (13)  [below left=0.2cm and -1.0cm of 12] {\verb|signed char|};
        \node (14)  [below left=0.2cm and -2.0cm of 13] {\verb|unsigned char|};
        \node (15)  [below right=0.8cm and -1.2cm of 9] {Знаковые};
        \node (16)  [below right=0.8cm and 1.5cm of 15] {\verb|int|};
        \node (17)  [below left=0.2cm and -0.8cm of 16] {\verb|short|};
        \node (18)  [below left=0.2cm and -0.6cm of 17] {\verb|long|};
        \node (19)  [below left=0.2cm and -1.0cm of 18] {\verb|long long|};
        \node (20)  [below left=0.8cm and -2.2cm of 9]  {Беззнаковые};
        \node (21)  [below right=0.8cm and 0.0cm of 20] {\verb|unsigned int|};
        \node (22)  [below left=0.2cm and -2.2cm of 21] {\verb|unsigned short|};
        \node (23)  [below left=0.2cm and -2.0cm of 22] {\verb|unsigned long|};
        \node (24)  [below left=0.2cm and -2.4cm of 23] {\verb|unsigned long long|};
        %
        \draw[->] (1)   -- (2);
        \draw[->] (1)   -- (3);
        \draw[->] (1)   -- (4);
        \draw[->] (4)   -- (5);
        \draw[->] (5)   -- (6);
        \draw[->] (5)   -- (7);
        \draw[->] (5)   -- (8.north west);
        \draw[->] (4)   -- (9);
        \draw[->] (9)   -- (10);
        \draw[->] (9)   -- (11.north west);
        \draw[->] (11)  -- (12.north west);
        \draw[->] (11)  -- (13.north west);
        \draw[->] (11)  -- (14.north west);
        \draw[->] (9)   -- (15);
        \draw[->] (15)  -- (16.north west);
        \draw[->] (15)  -- (17.north west);
        \draw[->] (15)  -- (18.north west);
        \draw[->] (15)  -- (19.north west);
        \draw[->] (9)   -- (20);
        \draw[->] (20)  -- (21.north west);
        \draw[->] (20)  -- (22.north west);
        \draw[->] (20)  -- (23.north west);
        \draw[->] (20)  -- (24.north west);
    \end{tikzpicture}

\end{frame}

\begin{frame}

    \frametitle{Источники и литаратура}

    \begin{itemize}

        \item \url{https://en.cppreference.com/w/cpp/language/type}

    \end{itemize}

\end{frame}

\end{document}
