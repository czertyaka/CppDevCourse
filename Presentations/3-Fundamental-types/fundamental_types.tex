% !TeX TS-program = xelatex

\documentclass[compress, 8pt]{beamer}

\usepackage{presentationtemplate}
\usepackage[askip=3mm, bskip=3mm]{terminal}
\usepackage[linenosfontsize=\tiny, askip=3mm, bskip=3mm]{mylisting}
\usepackage[dvipsnames, table]{xcolor}
\usepackage{tikz}
\usetikzlibrary{positioning}
\usepackage{array}
\usepackage{multirow}

\renewcommand{\arraystretch}{1.2}

\newenvironment{sixteenbit}{%
    \begin{center}
        \begin{tabular}{ |c|c|c|c|c|c|c|c|c|c|c|c|c|c|c|c|  }
            \hline
}{
        \hline
        \multicolumn{1}{c}{\color{gray}\tiny{15}} &
        \multicolumn{1}{c}{\color{gray}\tiny{14}} &
        \multicolumn{1}{c}{\color{gray}\tiny{13}} &
        \multicolumn{1}{c}{\color{gray}\tiny{12}} &
        \multicolumn{1}{c}{\color{gray}\tiny{11}} &
        \multicolumn{1}{c}{\color{gray}\tiny{10}} &
        \multicolumn{1}{c}{\color{gray}\tiny{9}} &
        \multicolumn{1}{c}{\color{gray}\tiny{8}} &
        \multicolumn{1}{c}{\color{gray}\tiny{7}} &
        \multicolumn{1}{c}{\color{gray}\tiny{6}} &
        \multicolumn{1}{c}{\color{gray}\tiny{5}} &
        \multicolumn{1}{c}{\color{gray}\tiny{4}} &
        \multicolumn{1}{c}{\color{gray}\tiny{3}} &
        \multicolumn{1}{c}{\color{gray}\tiny{2}} &
        \multicolumn{1}{c}{\color{gray}\tiny{1}} &
        \multicolumn{1}{c}{\color{gray}\tiny{0}} \\
        \end{tabular}
    \end{center}
}

\newenvironment{eightbit}{%
    \begin{center}
        \begin{tabular}{ |c|c|c|c|c|c|c|c|  }
            \hline
}{
        \hline
        \multicolumn{1}{c}{\color{gray}\tiny{7}} &
        \multicolumn{1}{c}{\color{gray}\tiny{6}} &
        \multicolumn{1}{c}{\color{gray}\tiny{5}} &
        \multicolumn{1}{c}{\color{gray}\tiny{4}} &
        \multicolumn{1}{c}{\color{gray}\tiny{3}} &
        \multicolumn{1}{c}{\color{gray}\tiny{2}} &
        \multicolumn{1}{c}{\color{gray}\tiny{1}} &
        \multicolumn{1}{c}{\color{gray}\tiny{0}} \\
        \end{tabular}
    \end{center}
}

\title{Фундаментальные типы данных}

\begin{document}

\frame[plain]{\titlepage}

\begin{frame}

    \frametitle{Что такое тип объекта?}

    Тип объекта
    \footnote{Помимо объектов типами также обладают
    ссылки, функции и выражения.}
    определяет:

    \begin{itemize}

        \item семантическое значение объекта, представленного
            набором бит в памяти;

        \item набор и реализацию операций, которые можно производить
            над объектом.

    \end{itemize}

\end{frame}

\begin{frame}[fragile]

    \frametitle{Семантика объекта}

    \myinputlisting[minted language=cpp]
        {Presentations/3-Fundamental-types/}
        {semantics.cpp}

    \begin{terminalwindow}
!\shellcommand{g++ semantics.cpp}!
!\shellcommand{./a.out}!
b=1, s=65, c=A
    \end{terminalwindow}

    Объекты, представленные одним и тем же набором бит в памяти, можно
    наделить разной семантикой: они будут по-разному интерпретироваться
    программистом, компилятором, средой выполнения и т.д.

\end{frame}

\begin{frame}[fragile]

    \frametitle{Набор операций над типом}

    \begin{columns}[T]

        \begin{column}{0.5\textwidth}

            \begin{myinplacelisting}[minted language=cpp]
int a = 0;
a + 1;
a * 2;

void foo() {}
// ...
foo();

float f = 0.1;
float* ptr = &f;
*ptr;
            \end{myinplacelisting}

            Для арифметических типов разрешены операции сложения
            и умножения, для функций разрешен оператор вызова,
            для указателей --- оператор разыменования и т.д.

        \end{column}

        \begin{column}{0.5\textwidth}

            Нельзя вызвать \verb|char| как функцию или разыменовать, как
            указатель, создать переменную с типом \verb|void| или
            применить оператор битового сдвига к переменной с типом
            плавающей точки.

            \begin{myinplacelisting}[minted language=cpp]
char c = 0;
c(); // compilation error
*c;  // compilation error

void v; // compilation error

float f;
f << 10; // compilation
         // error
            \end{myinplacelisting}

        \end{column}

    \end{columns}

\end{frame}

\begin{frame}[fragile]

    \frametitle{Набор операций над типом}

    \myinputlisting[minted language=cpp]
        {Presentations/3-Fundamental-types/}
        {indirection-on-int.cpp}

    \begin{terminalwindow}
!\shellcommand{g++ indirection-on-int.cpp}!
indirection-on-int.cpp: In function ‘int !\textcolor{teal}{main}!()’:
indirection-on-int.cpp:5:18: !\color{red}{error}!: invalid type argument of unary ‘*’ (have ‘int’)
    5 |     std::cout << !\textcolor{red}{*i}! << std::cout;
      |
    \end{terminalwindow}

    Допустимость проведения операции над объектом проверяется во время
    компиляции. Это прямое следствие того, что C++ --- статически
    типизированный язык.

\end{frame}

\begin{frame}[fragile]

    \frametitle{Различие в реализациях операций \\ над разными типами}

    \myinputlisting[minted language=cpp]
        {Presentations/3-Fundamental-types/}
        {sum.cpp}

    \begin{terminalwindow}
!\shellcommand{g++ -std=c++23 sum.cpp}!
    \end{terminalwindow}

    Выражения сумм \verb|u + u| и \verb|f + f| различаются семантикой, но не
    синтаксисом.

\end{frame}

\begin{frame}[fragile]

    \frametitle{Различие в реализациях операций \\ над разными типами}

    \verb|f = 2.0, u = 16384|

    \begin{sixteenbit}
        0 & \cellcolor{gray}{1} & 0 & 0 & 0 & 0 & 0 & 0 & 0 & 0 & 0 & 0 & 0 & 0 & 0 & 0 \\
    \end{sixteenbit}

    \verb|f + f = 4.0|

    \begin{sixteenbit}
        0 & \cellcolor{gray}{1} & 0 & 0 & 0 & \cellcolor{gray}{1} & 0 & 0 & 0 & 0 & 0 & 0 & 0 & 0 & 0 & 0 \\
    \end{sixteenbit}

    \verb|u + u = 32768|

    \begin{sixteenbit}
        \cellcolor{gray}{1} & 0 & 0 & 0 & 0 & 0 & 0 & 0 & 0 & 0 & 0 & 0 & 0 & 0 & 0 & 0 \\
    \end{sixteenbit}

\end{frame}

\begin{frame}[fragile]

    \frametitle{Классификация \\ фундаментальных типов}

    \begin{tikzpicture}
        \tikzstyle{every node}=[draw]
        %
        \node (1)                                       {Фундаментальные типы};
        \node (2)   [right=4.0cm of 1]                  {\verb|void|};
        \node (3)   [below right=0.1cm and 4.0cm of 1]  {\verb|std::nullptr_t|};
        \node (4)   [below=0.4cm of 1]                  {Арифметические типы};
        \node (5)   [below right=0.1cm and 1.5cm of 4]  {С плавающей точкой};
        \node (6)   [right=0.8cm of 5]                  {\verb|float|};
        \node (7)   [below=0.2cm of 6]                  {\verb|double|};
        \node (8)   [below left=0.2cm and -1.2cm of 7]  {\verb|long double|};
        \node (9)   [below=0.4cm of 4]                  {Целочисленные (интегральные)};
        \node (10)  [below right=0.1cm and 1.5cm of 9]  {\verb|bool|};
        \node (11)  [below right=0.8cm and 1.5cm of 9]  {Символьные};
        \node (12)  [below right=0.8cm and 1.5cm of 11] {\verb|char|};
        \node (13)  [below left=0.2cm and -1.0cm of 12] {\verb|signed char|};
        \node (14)  [below left=0.2cm and -2.0cm of 13] {\verb|unsigned char|};
        \node (15)  [below right=0.8cm and -1.2cm of 9] {Знаковые};
        \node (16)  [below right=0.8cm and 1.5cm of 15] {\verb|int|};
        \node (17)  [below left=0.2cm and -0.8cm of 16] {\verb|short|};
        \node (18)  [below left=0.2cm and -0.6cm of 17] {\verb|long|};
        \node (19)  [below left=0.2cm and -1.0cm of 18] {\verb|long long|};
        \node (20)  [below left=0.8cm and -2.2cm of 9]  {Беззнаковые};
        \node (21)  [below right=0.8cm and 0.0cm of 20] {\verb|unsigned int|};
        \node (22)  [below left=0.2cm and -2.2cm of 21] {\verb|unsigned short|};
        \node (23)  [below left=0.2cm and -2.0cm of 22] {\verb|unsigned long|};
        \node (24)  [below left=0.2cm and -2.4cm of 23] {\verb|unsigned long long|};
        %
        \draw[->] (1)   -- (2);
        \draw[->] (1)   -- (3);
        \draw[->] (1)   -- (4);
        \draw[->] (4)   -- (5);
        \draw[->] (5)   -- (6);
        \draw[->] (5)   -- (7);
        \draw[->] (5)   -- (8.north west);
        \draw[->] (4)   -- (9);
        \draw[->] (9)   -- (10);
        \draw[->] (9)   -- (11.north west);
        \draw[->] (11)  -- (12.north west);
        \draw[->] (11)  -- (13.north west);
        \draw[->] (11)  -- (14.north west);
        \draw[->] (9)   -- (15);
        \draw[->] (15)  -- (16.north west);
        \draw[->] (15)  -- (17.north west);
        \draw[->] (15)  -- (18.north west);
        \draw[->] (15)  -- (19.north west);
        \draw[->] (9)   -- (20);
        \draw[->] (20)  -- (21.north west);
        \draw[->] (20)  -- (22.north west);
        \draw[->] (20)  -- (23.north west);
        \draw[->] (20)  -- (24.north west);
    \end{tikzpicture}

\end{frame}

\begin{frame}[fragile]

    \frametitle{Тип \texttt{bool}}

    \begin{columns}[T]

        \begin{column}{0.5\textwidth}

            \verb|false|

            \begin{eightbit}
                0 & 0 & 0 & 0 & 0 & 0 & 0 & 0 \\
            \end{eightbit}

            \verb|true|

            \begin{eightbit}
                0 & 0 & 0 & 0 & 0 & 0 & 0 & \cellcolor{gray}{1} \\
            \end{eightbit}

        \end{column}

        \begin{column}{0.5\textwidth}

            Размер объекта типа \verb|bool| не определен стандартом,
            но на большинстве платформ равен 1 байту. \hfill \break

            \verb|true| и \verb|false| называются boolean literals
            \footnote{\tiny\url{https://en.cppreference.com/w/cpp/language/bool_literal}}.

        \end{column}

    \end{columns}

\end{frame}

\begin{frame}[fragile]

    \frametitle{Символьные типы}

    Таблица ASCII

    \scriptsize

    \begin{center}
        \begin{tabular}{|@{\space}m{0.30cm}m{0.35cm}|%
                         @{\space}m{0.30cm}m{0.35cm}|%
                         @{\space}m{0.30cm}m{0.35cm}|%
                         @{\space}m{0.30cm}m{0.35cm}|%
                         @{\space}m{0.30cm}m{0.35cm}|%
                         @{\space}m{0.30cm}m{0.35cm}|%
                         @{\space}m{0.30cm}m{0.35cm}|%
                         @{\space}m{0.30cm}m{0.35cm}|}
            \hline
            Hex & Char & Hex & Char & Hex & Char & Hex & Char &
            Hex & Char & Hex & Char & Hex & Char & Hex & Char \\
            \hline
            \hline
            \verb|00| & \verb|NUL| & \verb|10| & \verb|DLE| &
            \verb|20| & \verb|SP|  & \verb|30| & \verb|0|   &
            \verb|40| & \verb|@|   & \verb|50| & \verb|P|   &
            \verb|60| & \verb|`|   & \verb|70| & \verb|p|   \\
            \verb|01| & \verb|SOH| & \verb|11| & \verb|DC1| &
            \verb|21| & \verb|!|   & \verb|31| & \verb|1|   &
            \verb|41| & \verb|A|   & \verb|51| & \verb|Q|   &
            \verb|61| & \verb|a|   & \verb|71| & \verb|q|   \\
            \verb|02| & \verb|STX| & \verb|12| & \verb|DC2| &
            \verb|22| & \verb|"|   & \verb|32| & \verb|2|   &
            \verb|42| & \verb|B|   & \verb|52| & \verb|R|   &
            \verb|62| & \verb|b|   & \verb|72| & \verb|r|   \\
            \verb|03| & \verb|ETX| & \verb|13| & \verb|DC3| &
            \verb|23| & \verb|#|   & \verb|33| & \verb|3|   &
            \verb|43| & \verb|C|   & \verb|53| & \verb|S|   &
            \verb|63| & \verb|c|   & \verb|73| & \verb|s|   \\
            \verb|04| & \verb|EOT| & \verb|14| & \verb|DC4| &
            \verb|24| & \verb|$|   & \verb|34| & \verb|4|   &
            \verb|44| & \verb|D|   & \verb|54| & \verb|T|   &
            \verb|64| & \verb|d|   & \verb|74| & \verb|t|   \\
            \verb|05| & \verb|ENQ| & \verb|15| & \verb|NAK| &
            \verb|25| & \verb|%|   & \verb|35| & \verb|5|   &
            \verb|45| & \verb|E|   & \verb|55| & \verb|U|   &
            \verb|65| & \verb|e|   & \verb|75| & \verb|u|   \\
            \verb|06| & \verb|ACK| & \verb|16| & \verb|SYN| &
            \verb|26| & \verb|&|   & \verb|36| & \verb|6|   &
            \verb|46| & \verb|F|   & \verb|56| & \verb|V|   &
            \verb|66| & \verb|f|   & \verb|76| & \verb|v|   \\
            \verb|07| & \verb|BEL| & \verb|17| & \verb|ETB| &
            \verb|27| & \verb|'|   & \verb|37| & \verb|7|   &
            \verb|47| & \verb|G|   & \verb|57| & \verb|W|   &
            \verb|67| & \verb|g|   & \verb|77| & \verb|w|   \\
            \verb|08| & \verb|BS|  & \verb|18| & \verb|CAN| &
            \verb|28| & \verb|(|   & \verb|38| & \verb|8|   &
            \verb|48| & \verb|H|   & \verb|58| & \verb|X|   &
            \verb|68| & \verb|h|   & \verb|78| & \verb|x|   \\
            \verb|09| & \verb|HT|  & \verb|19| & \verb|EM|  &
            \verb|29| & \verb|)|   & \verb|39| & \verb|9|   &
            \verb|49| & \verb|I|   & \verb|59| & \verb|Y|   &
            \verb|69| & \verb|i|   & \verb|79| & \verb|y|   \\
            \verb|0A| & \verb|LF|  & \verb|1A| & \verb|SUB| &
            \verb|2A| & \verb|*|   & \verb|3A| & \verb|:|   &
            \verb|4A| & \verb|J|   & \verb|5A| & \verb|Z|   &
            \verb|6A| & \verb|j|   & \verb|7A| & \verb|z|   \\
            \verb|0B| & \verb|VT|  & \verb|1B| & \verb|ESC| &
            \verb|2B| & \verb|+|   & \verb|3B| & \verb|;|   &
            \verb|4B| & \verb|K|   & \verb|5B| & \verb|[|   &
            \verb|6B| & \verb|k|   & \verb|7B| & \verb|{|   \\
            \verb|0C| & \verb|FF|  & \verb|1C| & \verb|FS|  &
            \verb|2C| & \verb|,|   & \verb|3C| & \verb|<|   &
            \verb|4C| & \verb|L|   & \verb|5C| & \verb|\|   &
            \verb|6C| & \verb|l|   & \verb|7C| & \verb!|!   \\
            \verb|0D| & \verb|CR|  & \verb|1D| & \verb|GS|  &
            \verb|2D| & \verb|-|   & \verb|3D| & \verb|=|   &
            \verb|4D| & \verb|M|   & \verb|5D| & \verb|]|   &
            \verb|6D| & \verb|m|   & \verb|7D| & \verb|}|   \\
            \verb|0E| & \verb|SO|  & \verb|1E| & \verb|RS|  &
            \verb|2E| & \verb|.|   & \verb|3E| & \verb|>|   &
            \verb|4E| & \verb|N|   & \verb|5E| & \verb|^|   &
            \verb|6E| & \verb|n|   & \verb|7E| & \verb|~|   \\
            \verb|0E| & \verb|SI|  & \verb|1E| & \verb|US|  &
            \verb|2E| & \verb|/|   & \verb|3E| & \verb|?|   &
            \verb|4E| & \verb|O|   & \verb|5E| & \verb|_|   &
            \verb|6E| & \verb|o|   & \verb|7E| & \verb|DEL| \\
            \hline
        \end{tabular}
    \end{center}

\end{frame}

\begin{frame}[fragile]

    \frametitle{Символьные типы}

    \begin{itemize}
        \item \verb|signed char| --- знаковый тип представления символа.
        \item \verb|unsigned char| --- беззнаковый тип представления символа.
        \item \verb|char| --- тип представления символа, знаковость зависит от
            компилятора и целевой платформы.
        \item \verb|wchar_t|, \verb|char16_t|, ...
    \end{itemize}

    \hfill \break
    Стандарт C++ гарантирует, что \verb|1 == sizeof(char)|.

\end{frame}

\begin{frame}[fragile]

    \frametitle{Символьные типы}

    \begin{center}
        \begin{tabular}{|c|c|c|}
            \hline
            Тип & Размер & Диапазон значений \\
            \hline
            \hline
            \verb|signed char| & 1 & \([-128_{10}, 127_{10}]\) \\
            \verb|unsigned char| & 1 & \([0, 255_{10}]\) \\
            \verb|char| & 1 & возможны оба варианта \\
            \hline
        \end{tabular}
    \end{center}

\end{frame}

\begin{frame}[fragile]

    \frametitle{Символьные типы}

    \myinputlisting[minted language=cpp]
        {Presentations/3-Fundamental-types/}
        {char.cpp}

    \begin{terminalwindow}
!\shellcommand{g++ char.cpp}!
!\shellcommand{./a.out}!
A
b
this is tab     and this '
' is newline
    \end{terminalwindow}

\end{frame}

\begin{frame}[fragile]

    \frametitle{Остальные интегральные типы}

    Базовым интегральным типом является \verb|int|.
    Другие интегральные типы образуются комбинацией ключевого слова \verb|int| с модификаторами
    \verb|signed|, \verb|unsigned|, \verb|short| и \verb|long|.
    При комбинации с модификаторами допускается пропускать ключевое слово \verb|int|.

    \hfill \break
    Примеры допустимых типов:

    \begin{itemize}
        \item \verb|short int|
        \item \verb|int long|
        \item \verb|short|
        \item \verb|signed long long|
        \item \verb|long unsigned|
        \item ...
    \end{itemize}

\end{frame}

\begin{frame}[fragile]

    \frametitle{Размеры интегральных типов}

    \begin{center}
        \begin{tabular}{|c|c|}
            \hline
            Минимальное количество бит & Типы \\
            \hline
            \hline
            \multirow{4}{4em}{16} & \verb|int| \\
            & \verb|unsigned int| \\
            & \verb|short int| \\
            & \verb|unsigned short int| \\
            \hline
            \multirow{2}{4em}{32} & \verb|long int| \\
            & \verb|unsigned long int| \\
            \hline
            \multirow{2}{4em}{64} & \verb|long long int| \\
            & \verb|unsigned long long int| \\
            \hline
        \end{tabular}
    \end{center}

    \hfill \break
    Стандарт C++ гарантирует, что

    \begin{center}
        \verb|sizeof(char) ≤ sizeof(short) ≤ sizeof(int)|
        \verb|≤ sizeof(long) ≤ sizeof(long long)|
    \end{center}

\end{frame}

\begin{frame}[fragile]

    \frametitle{Представление знаковых \\ интегральных типов}

\end{frame}

\begin{frame}[fragile]

    \frametitle{Представление беззнаковых \\ интегральных типов}

\end{frame}

\begin{frame}

    \frametitle{Источники и литература}

    \begin{itemize}

        \item \href{https://en.cppreference.com/w/cpp/language/type}
            {Классификация типов на cppreference}
        \item \href{https://habr.com/ru/companies/wunderfund/articles/777850/}
            {Статья про Unicode}
        \item \href{https://github.com/Nekrolm/ubbook/blob/master/numeric/overflow.md}
            {Переполнение целых знаковых типов}

    \end{itemize}

\end{frame}

\end{document}
