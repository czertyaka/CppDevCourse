% !TeX TS-program = xelatex

\documentclass[compress, 8pt]{beamer}

\usepackage{presentationtemplate}
\usepackage[askip=3mm, bskip=3mm]{terminal}
\usepackage[linenosfontsize=\tiny, askip=3mm, bskip=3mm]{mylisting}
\usepackage{tcolorbox}

\newtcolorbox{task}{
    colback=yellow!50!white,
    boxrule=0.02cm,
    colframe=black,
    sharp corners,
    left=0mm,
    right=0mm,
    top=0mm,
    bottom=0mm,
    before upper={\textbf{Задание}:\:},
}

\title{Условия}

\begin{document}

\frame[plain]{\titlepage}

\begin{frame}[fragile]

    \frametitle{Selection statements}

    Операторы условий нужны для создания ветвлений в коде: выполнения тех или
    иных частей программы на основе информации, доступной во время её выполнения.
    В C++ имеется три selection statements:

    \hfill \break

    \begin{itemize}
        \item оператор \verb|if|\footnotemark{}
        \begin{myinplacelisting}[minted language=cpp]
if (/*условие*/) {/*тело условия*/}
        \end{myinplacelisting}

        \item оператор \verb|if-else|
        \begin{myinplacelisting}[minted language=cpp]
if (/*условие*/) {/*тело условия*/} else {/*тело условия*/}
        \end{myinplacelisting}

        \item оператор \verb|switch|
        \begin{myinplacelisting}[minted language=cpp]
switch (/*условие*/) {/*тело условия*/}
        \end{myinplacelisting}
    \end{itemize}

    \footnotetext{Здесь и далее обращение с терминологией может быть несколько
    вольным и может не всегда соотвествовать слову стандарта. Так, “оператор if”
    в этом контексте соответсвует \texttt{if~statement} в стандарте языка.}

\end{frame}

\begin{frame}[fragile]

    \frametitle{Оператор \texttt{if}}

    Простейший пример использования оператора \verb|if|:

    \begin{myinplacelisting}[minted language=cpp]
bool get_input_from_user();
void do_something();

int main() {

    const bool condition = get_input_from_user();
    if (condition) {    // если condition равно true,
        do_something(); // то выполняем функцию do_something
    }

    // или

    if (get_input_from_user()) {
        do_something();
    }

}
    \end{myinplacelisting}

\end{frame}

\begin{frame}[fragile]

    \frametitle{Оператор \texttt{if}}

    Поместить одну инструкцию (statement) в тело условия можно
    без фигурных скобок:

    \begin{myinplacelisting}[minted language=cpp]
bool get_input_from_user();
void do_something();
void do_something_else();

int main() {
    const bool condition = get_input_from_user();
    if (condition)
        do_something();
        do_something_else();

    // эквивалентно

    if (condition) {
        do_something();
    }
    do_something_else();
}
    \end{myinplacelisting}

\end{frame}

\begin{frame}[fragile]

    \frametitle{Оператор \texttt{if}}

    Использовать тело условия без явного указания его
    границ может быть опасно.

    \begin{task}
        Найдите проблемы в коде на листинге
    \end{task}

    \begin{myinplacelisting}[minted language=cpp]
#include <iostream>

#define PRINT_SQUARE(X)                 \
    auto square = X * X;                \
    std::cout << "square of " << X      \
        << " = " << square << std::endl;

int main() {
    int x = 2;
    if (true)
        PRINT_SQUARE(x);
}
    \end{myinplacelisting}

\end{frame}

\begin{frame}[fragile]

    \frametitle{Оператор \texttt{if-else}}

    Простейший пример использования оператора \verb|if-else|:

    \begin{myinplacelisting}[minted language=cpp]
bool get_input_from_user();
void do_something();
void do_something_else();

int main() {
    const bool condition = get_input_from_user();
    if (condition) {
        do_something();
    }
    else {                      // если condition равно
                                // false,
        do_something_else();    // то выполняем функцию
                                // do_something_else
    }
}
    \end{myinplacelisting}

\end{frame}

\begin{frame}[fragile]

    \frametitle{Оператор \texttt{if-else}}

    Как и в случае с оператором \verb|if|, ограничивать
    тела условий в \verb|if-else| фигурными скобками необязательно.
    Но это может привести к нежелательным последствиям.

    \begin{task}
        Найдите проблемы в коде на листинге
    \end{task}

    \begin{myinplacelisting}[minted language=cpp]
#include <iostream>

bool user_authenticated();
bool user_is_admin();

int main() {
    if (user_authenticated())
        if (user_is_admin())
            std::cout << "user is admin\n";
    else
        std::cout << "user is not authenticated\n";
}
    \end{myinplacelisting}

\end{frame}

\begin{frame}[fragile]

    \frametitle{Оператор \texttt{if-else}}

    С помощью оператора \verb|if-else| можно проверить несколько условий.

    \begin{myinplacelisting}[minted language=cpp]
#include <iostream>

bool user_not_authenticated();
bool user_is_admin();

int main() {
    if (user_not_authenticated()) {
        std::cout << "user is not authenticated\n";
    }
    else if (user_is_admin()) {
        std::cout << "user is admin\n";
    }
    else {
        std::cout << "regular user\n";
    }
}
    \end{myinplacelisting}

\end{frame}

\begin{frame}[fragile]

    \frametitle{Операторы сравнения}

    Для арифметических типов и указателей опеределены операторы сравнения.
    Операторы сравнения бинарны: из двух объектов арифметических типов образуют
    выражение, возвращающее объект типа \verb|bool|.

    \hfill\break
    В C++ существуют следующие операторы сравнения
    \footnote{За исключением оператора
        \href{https://en.cppreference.com/w/cpp/language/operator\_comparison}{трехстороннего сравнения}}
    :

    \begin{itemize}
        \item строгое и нестрогое \textbf{БОЛЬШЕ} \verb|>| и \verb|>=|;
        \item строгое и нестрогое \textbf{МЕНЬШЕ} \verb|<| и \verb|<=|;
        \item \textbf{РАВНО} \verb|==|;
        \item \textbf{НЕ РАВНО} \verb|!=|.
    \end{itemize}

\end{frame}

\begin{frame}[fragile]

    \frametitle{Операторы сравнения}

    Примеры использования операторов сравнения:

    \begin{myinplacelisting}[minted language=cpp]
bool a = 3 >= 3;    // true
bool b = 3.0 > 2.0; // true

bool c = (5 == 6);  // false
bool d = 2 < 1;     // false

int get_x();
int get_y();

if (get_x() >= 0) {
    std::cout << "x is greater than 0\n";
}
else if (get_x() == get_y()) {
    std::cout << "x == y, both are < 0\n";
}
    \end{myinplacelisting}

\end{frame}

\begin{frame}[fragile]

    \frametitle{Операторы сравнения}

    \begin{task}
        Найдите логическую ошибку в коде на листинге
    \end{task}

    \begin{myinplacelisting}[minted language=cpp]
#include <iostream>

int get_score();

int main() {
    const int score = get_score();
    if (score >= 40) {
        std::cout << "satisfactory\n";
    }
    else if (score >= 60) {
        std::cout << "good\n";
    }
    else if (score >= 80) {
        std::cout << "excellent\n";
    }
}
    \end{myinplacelisting}

\end{frame}

\begin{frame}[fragile]

    \frametitle{Логические операторы}

    Логические операторы позволяют создавать составные условия из
    других условий.

    \hfill \break

    В C++ существует три логических оператора:

    \begin{itemize}
        \item логическое \textbf{ИЛИ} \verb!||!;
        \item логическое \textbf{И} \verb|&&|;
        \item логическое \textbf{НЕ} \verb|!|.
    \end{itemize}

\end{frame}

\begin{frame}[fragile]

    \frametitle{Логические операторы}

    Пример использования логических операторов:

    \begin{myinplacelisting}[%
        minted language=cpp,
        minted options={
            fontsize=\normalsize,
            breaklines,
            breakanywhere,
            linenos,
            escapeinside=??,
            numbersep=2pt
        },
    ]
#include <iostream>

int get_number_from_user();

int main {
    int x = get_x();
    if (x != 7 && x != 8) {
        std::cout << "x is neither seven nor eight!\n";
    }
    if (x == 7 || x == 8 || x == 9) {
        std::cout << "x is either seven or eight or"
            << " nine!\n";
    }
    if (!(x != 7 && x != 8)) {
        std::cout << "x is either seven or eight\n";
    }
}
\end{myinplacelisting}

\end{frame}

\begin{frame}[fragile]

    \frametitle{Логические операторы}

    \begin{task}
        Объясните, в чем ошибка использования логических операторов
        в коде на листинге
    \end{task}

    \begin{myinplacelisting}[%
        minted language=cpp,
        minted options={
            fontsize=\normalsize,
            breaklines,
            breakanywhere,
            linenos,
            escapeinside=??,
            numbersep=2pt
        },
    ]
int a = get_a();
int b = get_b();
a < b && b < a;
a < b || b < a;
!(a < b) || (b > a);

unsigned x = get_x();
x >= 0 && x < 3;
    \end{myinplacelisting}

\end{frame}

\begin{frame}[fragile]

    \frametitle{Оператор \texttt{switch}}

    Представим, что нам необходимо написать очень простую программу анализа
    ответа на HTTP-запрос.
    В случае, если код ответа равен 200, программа должна вывести сообщение
    об успешности обработки отправленного запроса, для кода ответа, равного
    404, — о том, что запрашиваемый ресурс не был найден, а для 500 — о
    внутренней ошибке сервера
    \footnote{Полный список всех кодов статуса HTTP:
        \url{https://developer.mozilla.org/en-US/docs/Web/HTTP/Status}}.

\end{frame}

\begin{frame}[fragile]

    \frametitle{Оператор \texttt{switch}}

    \begin{myinplacelisting}[minted language=cpp]
#include <iostream>

#define HTTP_OK 200
#define HTTP_NOT_FOUND 404
#define HTTP_INTERNAL_SERVER_ERROR 500

void process_http_status_code(const int code) {
    if (code == HTTP_OK) {
        std::cout << "success" << std::endl;
    }
    else if (code == HTTP_NOT_FOUND) {
        std::cout << "not found" << std::endl;
    }
    else if (code == HTTP_INTERNAL_SERVER_ERROR) {
        std::cout << "internal server error" << std::endl;
    }
    else {
        std::cout << "unknown code: " << code << std::endl;
    }
}
    \end{myinplacelisting}

\end{frame}

\begin{frame}[fragile]

    \frametitle{Оператор \texttt{switch}}

    \hfill \break

    Для создания ветвления на основе результата вычисления выражения,
    возвращающего значение интегрального типа, можно использовать
    оператор \verb|switch|.

    \begin{myinplacelisting}[%
        minted language=cpp,
        minted options={
            fontsize=\footnotesize,
            breaklines,
            breakanywhere,
            linenos,
            escapeinside=||,
            numbersep=2pt
        },
    ]
#include <iostream>

#define HTTP_OK 200
#define HTTP_NOT_FOUND 404
#define HTTP_INTERNAL_SERVER_ERROR 500

void process_http_status_code(const int code) {
    switch(code) {
        case HTTP_OK:
            std::cout << "success" << std::endl;
            break;
        case HTTP_NOT_FOUND:
            std::cout << "not found" << std::endl;
            break;
        case HTTP_INTERNAL_SERVER_ERROR:
            std::cout << "internal server error" << std::endl;
            break;
        default:
            std::cout << "unknown code: " << code << std::endl;
            break;
    }
}
    \end{myinplacelisting}

\end{frame}

\begin{frame}[fragile]

    \frametitle{Оператор \texttt{switch}}

    \hfill \break

    В случае удачной проверки одного из \verb|case|, выполнится не только
    код этого \verb|case|, но и код в остальных \verb|case|, которые
    расположены ниже текущего.
    Эта особенность в английских источниках называется \textbf{fallthrough}.

    \begin{myinplacelisting}[%
        minted language=cpp,
        minted options={
            fontsize=\footnotesize,
            breaklines,
            breakanywhere,
            linenos,
            escapeinside=||,
            numbersep=2pt
        },
    ]
#include <iostream>

#define HTTP_OK 200
#define HTTP_NOT_FOUND 404
#define HTTP_INTERNAL_SERVER_ERROR 500
#define HTTP_SERVICE_UNAVALABLE 503

void process_http_status_code(const int code) {
    switch(code) {
        case HTTP_OK:
            std::cout << "success" << std::endl;
            break;
        case HTTP_NOT_FOUND:
            std::cout << "not found" << std::endl;
            break;
        case HTTP_INTERNAL_SERVER_ERROR:
        case HTTP_SERVICE_UNAVAILABLE:
            std::cout << "server error" << std::endl;
            break;
        default:
            std::cout << "unknown code: " << code << std::endl;
            break;
    }
}
    \end{myinplacelisting}

\end{frame}

\end{document}
