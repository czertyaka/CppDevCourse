% !TeX TS-program = xelatex

\documentclass[compress]{beamer}

\beamertemplatenavigationsymbolsempty
\setbeamertemplate{headline}{}

% ==== Packages section ============================

\usepackage{xltxtra}
\usepackage[main=russian,english]{babel}

\setmainfont{DejaVu Serif}
\setsansfont{DejaVu Sans}
\setmonofont{Fira Mono}

\usepackage[export]{adjustbox}
\usepackage{tikz}
\usetikzlibrary{matrix}

\usepackage[newfloat]{minted}
\usemintedstyle{borland}
\SetupFloatingEnvironment{listing}{name=Листинг}

\usepackage[font=tiny,skip=0pt]{caption}

\usepackage{fancyvrb}
\usepackage{fvextra}

\usepackage{hyperref}
\hypersetup{
    colorlinks=true,
    linkcolor=blue,
    filecolor=magenta,
    urlcolor=cyan,
}

\usepackage{tikz}
\usetikzlibrary{positioning}

\usepackage{makecell}

% ==== Beamer customization section ================

\title{Фазы трансляции}
\subtitle{Введение в разработку на C++}
\author{Пыхов~Олег~Александрович\inst{1} \\ \texttt{pyhovoleg31@gmail.com}}
\institute{
    \inst{1}%
    ООО~"Прософт-Системы"
}
\date{2024}
\titlegraphic{
    \includegraphics[scale=0.1,valign=t]{Presentations/images/university-logo.png}
    \includegraphics[scale=0.3,valign=t]{Presentations/images/prosoft-logo.png}
}
\logo{
    \begin{tikzpicture}[overlay,remember picture]
        \node[left=0.2cm] at (current page.32){
            \includegraphics[scale=0.15,valign=t]{Presentations/images/prosoft-logo.png}
            \includegraphics[scale=0.05,valign=t]{Presentations/images/university-logo.png}
        };
    \end{tikzpicture}
}

\setbeamertemplate{footline}[frame number]

\begin{document}

\frame[plain]{\titlepage}

\begin{frame}

    \frametitle{Фазы трансляции}

    \begin{enumerate}

        \item Mapping source characters

        \item Splicing lines

        \item Lexing

        \item \textbf{Preprocessing (препроцессор)}

        \item Determining common string literal encodings

        \item Concatenating string literals

        \item \textbf{Compiling (компиляция)}

        \item Instantiating templates

        \item \textbf{Linking (линковка)}

    \end{enumerate}

    \vfill

    Более подробно:
    \href{https://en.cppreference.com/w/cpp/language/translation_phases}{eng},
    \href{https://ru.cppreference.com/w/cpp/language/translation_phases}{ru}.

\end{frame}

\begin{frame}

    \frametitle{Компиляторы C++}

    \begin{itemize}

        \item \texttt{\textbf{g++}} от GNU Compiler Collection (GCC)

        \item \texttt{\textbf{clang++}} от от Low Level Virtual Machine (LLVM)

        \item Microsoft Visual C++ (MSVC)

        \item и др.

    \end{itemize}

\end{frame}

\begin{frame}

    \frametitle{Препроцессор}

    \centering

    \begin{tikzpicture}[node distance = 5cm]

        \node (source1) {
            \includegraphics[width=1.7cm]{Presentations/images/source_code.png}
        };
        \node (source2) [right=of source1] {
            \includegraphics[width=1.7cm]{Presentations/images/source_code.png}
        };

        \draw[->, ultra thick] (source1.east) -- (source2.west)
            node[midway, above] {\makecell[l]{Работа\\препроцессора}};

    \end{tikzpicture}

\end{frame}

\begin{frame}[fragile]

    \frametitle{Препроцессор: директива \\ \texttt{\#define}}

    \begin{columns}[T]

        \begin{column}{0.5\textwidth}

            \begin{listing}[H]
                \inputminted[
                    frame=single,
                    fontsize=\scriptsize,
                    linenos,
                    xleftmargin=1em
                ]{cpp}
                    {Presentations/2-Phases-of-translation/defines-example.cpp}
                \caption{defines-example.cpp}
            \end{listing}

        \end{column}

        \begin{column}{0.5\textwidth}

            \begin{Verbatim}[
                frame=single,
                fontsize=\tiny,
                commandchars=\\\{\},
                breaklines=true,
                breakanywhere=true,
            ]
$ \textbf{g++ \colorbox{yellow}{-E} defines-example.cpp}
# 0 "defines-example.cpp"
# 0 "<built-in>"
# 0 "<command-line>"
# 1 "/usr/include/stdc-predef.h" 1 3 4
# 0 "<command-line>" 2
# 1 "defines-example.cpp"




int main() \{
    int i = 1 + 32;
    return 0;
\}
            \end{Verbatim}

        \end{column}

    \end{columns}

\end{frame}

\begin{frame}[fragile]

    \frametitle{Препроцессор: директива \\ \texttt{\#define}}

    \begin{columns}[T]

        \begin{column}{0.5\textwidth}

            \begin{listing}[H]
                \inputminted[
                    frame=single,
                    fontsize=\scriptsize,
                    linenos,
                    xleftmargin=1em
                ]{cpp}
                    {Presentations/2-Phases-of-translation/defines-param-example.cpp}
                \caption{define-parameters.cpp}
            \end{listing}

        \end{column}

        \begin{column}{0.5\textwidth}

            \begin{Verbatim}[
                frame=single,
                fontsize=\scriptsize,
                commandchars=\\\{\},
                breaklines=true,
                breakanywhere=true,
            ]
$ \textbf{g++ define-parameters.cpp}
$ \textbf{./a.out}
1, 3.14, text
            \end{Verbatim}

        \end{column}

    \end{columns}

\end{frame}

\begin{frame}[fragile]

    \frametitle{Препроцессор: директива \\ \texttt{\#include}}

    \begin{columns}[T]

        \begin{column}{0.5\textwidth}

            \begin{listing}[H]
                \inputminted[
                    frame=single,
                    fontsize=\scriptsize,
                    linenos,
                    xleftmargin=1em
                ]{cpp}
                    {Presentations/2-Phases-of-translation/include-header.h}
                \caption{include-header.h}
            \end{listing}

            \begin{listing}[H]
                \inputminted[
                    frame=single,
                    fontsize=\scriptsize,
                    linenos,
                    xleftmargin=1em
                ]{cpp}
                    {Presentations/2-Phases-of-translation/include-source.cpp}
                \caption{include-source.cpp}
            \end{listing}

        \end{column}

        \begin{column}{0.5\textwidth}

            \begin{Verbatim}[
                frame=single,
                fontsize=\tiny,
                commandchars=\\\{\},
                breaklines=true,
                breakanywhere=true,
            ]
$ \textbf{g++ -E include-source.cpp}
# 0 "include-source.cpp"
# 0 "<built-in>"
# 0 "<command-line>"
# 1 "/usr/include/stdc-predef.h" 1 3 4
# 0 "<command-line>" 2
# 1 "include-source.cpp"
# 1 "include-header.h" 1
\colorbox{yellow}{int a = 0;}
# 2 "include-source.cpp" 2

int main() \{
    int b = a;
    return 0;
\}
            \end{Verbatim}

        \end{column}

    \end{columns}

\end{frame}

\begin{frame}[fragile]

    \frametitle{Препроцессор: директива \\ \texttt{\#pragma}}

    \begin{columns}[T]

        \begin{column}{0.5\textwidth}

            \begin{listing}[H]
                \inputminted[
                    frame=single,
                    fontsize=\scriptsize,
                    linenos,
                    xleftmargin=1em
                ]{cpp}
                    {Presentations/2-Phases-of-translation/pragma-example.h}
                \caption{pragma-example.h}
            \end{listing}

            \begin{listing}[H]
                \inputminted[
                    frame=single,
                    fontsize=\scriptsize,
                    linenos,
                    xleftmargin=1em
                ]{cpp}
                    {Presentations/2-Phases-of-translation/pragma-example.cpp}
                \caption{pragma-example.cpp}
            \end{listing}

        \end{column}

        \begin{column}{0.5\textwidth}

            \begin{Verbatim}[
                frame=single,
                fontsize=\tiny,
                commandchars=\\\{\},
                breaklines=true,
                breakanywhere=true,
            ]
$ \textbf{g++ -E pragma-example.cpp}
# 0 "pragma-example.cpp"
# 0 "<built-in>"
# 0 "<command-line>"
# 1 "/usr/include/stdc-predef.h" 1 3 4
# 0 "<command-line>" 2
# 1 "pragma-example.cpp"
# 1 "pragma-example.h" 1


const auto PI = 3.14;
# 2 "pragma-example.cpp" 2




int main() \{
    auto pi = PI;
\}
            \end{Verbatim}

        \end{column}

    \end{columns}

\end{frame}

\begin{frame}

    \frametitle{Компиляция}

    \centering

    \begin{tikzpicture}[node distance = 5cm]

        \node (source) {
            \includegraphics[width=1.7cm]{Presentations/images/source_code.png}
        };
        \node (binary) [right=of source] {
            \includegraphics[width=1.7cm]{Presentations/images/binary-file.png}
        };

        \draw[->, ultra thick] (source.east) -- (binary.west)
            node[midway, above] {Компиляция};

    \end{tikzpicture}

\end{frame}

\begin{frame}[fragile]

    \frametitle{Компиляция}

    \begin{columns}[T]

        \begin{column}{0.25\textwidth}

            \begin{listing}[H]
                \inputminted[
                    frame=single,
                    fontsize=\tiny,
                    linenos,
                    xleftmargin=1em
                ]{cpp}
                    {Presentations/2-Phases-of-translation/compilation-example.cpp}
                \caption{compilation-example.cpp}
            \end{listing}

        \end{column}

        \begin{column}{0.75\textwidth}

            \begin{Verbatim}[
                frame=single,
                fontsize=\tiny,
                commandchars=\\\{\},
                breaklines=true,
                breakanywhere=true,
            ]
$ \textbf{g++ \colorbox{yellow}{-c} compilation-example.cpp -o compilation-example.o}
$ \textbf{objdump -f compilation-example.o}

compilation-example.o:     file format \colorbox{green}{elf64-x86-64}
architecture: i386:x86-64, flags 0x00000011:
HAS_RELOC, HAS_SYMS
start address 0x0000000000000000
$ \textbf{objdump \colorbox{yellow}{-tC} compilation-example.o}

compilation-example.o:     file format elf64-x86-64

SYMBOL TABLE:
0000000000000000 l    df *ABS*  0000000000000000 compilation-example.cpp
0000000000000000 l    d  .text  0000000000000000 .text
0000000000000000 l    d  .rodata        0000000000000000 .rodata
0000000000000000 g     F .text  000000000000000e \colorbox{green}{PI()}
000000000000000e g     F .text  000000000000001b \colorbox{green}{main}
            \end{Verbatim}

        \end{column}

    \end{columns}

\end{frame}

\begin{frame}[fragile]

    \frametitle{Компиляция}

        \begin{Verbatim}[
            frame=single,
            fontsize=\tiny,
            commandchars=\\\{\},
            breaklines=true,
            breakanywhere=true,
        ]
$ \textbf{objdump \colorbox{yellow}{-d} compilation-example.o}

compilation-example.o:     file format elf64-x86-64


Disassembly of section .text:

0000000000000000 <_Z2PIv>:
   0:   55                      push   %rbp
   1:   48 89 e5                mov    %rsp,%rbp
   4:   f3 0f 10 05 00 00 00    movss  0x0(%rip),%xmm0        # c <_Z2PIv+0xc>
   b:   00
   c:   5d                      pop    %rbp
   d:   c3                      ret

000000000000000e <main>:
   e:   55                      push   %rbp
   f:   48 89 e5                mov    %rsp,%rbp
  12:   48 83 ec 10             sub    $0x10,%rsp
  16:   e8 00 00 00 00          call   1b <main+0xd>
  1b:   66 0f 7e c0             movd   %xmm0,%eax
  1f:   89 45 fc                mov    %eax,-0x4(%rbp)
  22:   b8 00 00 00 00          mov    $0x0,%eax
  27:   c9                      leave
  28:   c3                      ret
        \end{Verbatim}

    \centering

\end{frame}

\begin{frame}[fragile]

    \frametitle{Линковка}

    \centering

    \begin{tikzpicture}[node distance=5cm]

        \matrix (objs) [row sep=0.1cm, column sep=0.1cm] {
            \node (bin1) {
                \includegraphics[width=1.5cm]{Presentations/images/binary-file.png}
            }; &
            \node (bin2) {
                \includegraphics[width=1.5cm]{Presentations/images/binary-file.png}
            }; \\
            \node (bin3) {
                \includegraphics[width=1.5cm]{Presentations/images/binary-file.png}
            }; &
            \node (bin4) {
                \includegraphics[width=1.5cm]{Presentations/images/binary-file.png}
            }; \\
        };

        \node (exe) [right=of objs] {
            \includegraphics[width=1.5cm]{Presentations/images/binary-file.png}
        };

        \draw[->, ultra thick] (objs.east) -- (exe.west)
            node[midway, above] {Линковка};

    \end{tikzpicture}

\end{frame}

\begin{frame}[fragile]

    \frametitle{Линковка}

    \begin{columns}[T]

        \begin{column}{0.4\textwidth}

            \begin{listing}[H]
                \inputminted[
                    frame=single,
                    fontsize=\tiny,
                    linenos,
                    xleftmargin=1em
                ]{cpp}
                    {Presentations/2-Phases-of-translation/link-example.h}
                \caption{link-example.h}
            \end{listing}

            \begin{listing}[H]
                \inputminted[
                    frame=single,
                    fontsize=\tiny,
                    linenos,
                    xleftmargin=1em
                ]{cpp}
                    {Presentations/2-Phases-of-translation/link-example.cpp}
                \caption{link-example.cpp}
            \end{listing}

            \begin{listing}[H]
                \inputminted[
                    frame=single,
                    fontsize=\tiny,
                    linenos,
                    xleftmargin=1em
                ]{cpp}
                    {Presentations/2-Phases-of-translation/link-example-main.cpp}
                \caption{link-example-main.cpp}
            \end{listing}

        \end{column}

        \begin{column}{0.6\textwidth}

            \begin{Verbatim}[
                frame=single,
                fontsize=\tiny,
                commandchars=\\\{\},
                breaklines=true,
                breakanywhere=true,
            ]
$ \textbf{g++ -c link-example.cpp}
$ \textbf{g++ -c link-example-main.cpp}
$ \textbf{g++ link-example.o link-example-main.o}
$ \textbf{./a.out}
Hello!
$ \textbf{file a.out}
a.out: ELF \colorbox{green}{64-bit} LSB executable, \colorbox{green}{x86-64}, version 1 (SYSV), dynamically linked, interpreter /lib64/ld-linux-x86-64.so.2, BuildID[sha1]=9e6a23c78b72c21d9e51a783409c9ec5064a8d55, for GNU/Linux 3.2.0, not stripped
            \end{Verbatim}

        \end{column}

    \end{columns}

\end{frame}

\end{document}
