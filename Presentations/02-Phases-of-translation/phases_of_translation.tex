% !TeX TS-program = xelatex

\documentclass[compress]{beamer}

\usepackage{presentationtemplate}
\usepackage[fontsize=\tiny, askip=3mm]{terminal}
\usepackage[fontsize=\tiny, linenosfontsize=\tiny, askip=3mm]{mylisting}
\usepackage{makecell}
\usepackage{tikz}

% tikz libraries
\usetikzlibrary{matrix}
\usetikzlibrary{positioning}

\title{Фазы трансляции}

\begin{document}

\frame[plain]{\titlepage}

\begin{frame}

    \frametitle{Фазы трансляции}

    \begin{enumerate}

        \item Mapping source characters

        \item Splicing lines

        \item Lexing

        \item \textbf{Preprocessing (препроцессор)}

        \item Determining common string literal encodings

        \item Concatenating string literals

        \item \textbf{Compiling (компиляция)}

        \item Instantiating templates

        \item \textbf{Linking (линковка)}

    \end{enumerate}

    \vfill

    Более подробно:
    \href{https://en.cppreference.com/w/cpp/language/translation_phases}{eng},
    \href{https://ru.cppreference.com/w/cpp/language/translation_phases}{ru}.

\end{frame}

\begin{frame}

    \frametitle{Компиляторы C++}

    \begin{itemize}

        \item \texttt{\textbf{g++}} от GNU Compiler Collection (GCC)

        \item \texttt{\textbf{clang++}} от от Low Level Virtual Machine (LLVM)

        \item Microsoft Visual C++ (MSVC)

        \item и др.

    \end{itemize}

\end{frame}

\begin{frame}

    \frametitle{Препроцессор}

    \centering

    \begin{tikzpicture}[node distance = 5cm]

        \node (source1) {
            \includegraphics[width=1.7cm]{Presentations/images/source_code.png}
        };
        \node (source2) [right=of source1] {
            \includegraphics[width=1.7cm]{Presentations/images/source_code.png}
        };

        \draw[->, ultra thick] (source1.east) -- (source2.west)
            node[midway, above] {\makecell[l]{Работа\\препроцессора}};

    \end{tikzpicture}

\end{frame}

\begin{frame}[fragile]

    \frametitle{Препроцессор: директива \\ \texttt{\#define}}

    \begin{columns}[T]

        \begin{column}{0.5\textwidth}

            \myinputlisting[minted language=cpp]
                {Presentations/02-Phases-of-translation/}
                {defines-example.cpp}

        \end{column}

        \begin{column}{0.5\textwidth}

            \begin{terminalwindow}
!\shellcommand{g++ \colorbox{yellow}{-E} defines-example.cpp}!
# 0 "defines-example.cpp"
# 0 "<built-in>"
# 0 "<command-line>"
# 1 "/usr/include/stdc-predef.h" 1 3 4
# 0 "<command-line>" 2
# 1 "defines-example.cpp"




int main() {
    int i = 1 + 32;
    return 0;
}
            \end{terminalwindow}

        \end{column}

    \end{columns}

\end{frame}

\begin{frame}[fragile]

    \frametitle{Препроцессор: директива \\ \texttt{\#define}}

    \begin{columns}[T]

        \begin{column}{0.5\textwidth}

            \myinputlisting[minted language=cpp]
                {Presentations/02-Phases-of-translation/}
                {defines-param-example.cpp}

        \end{column}

        \begin{column}{0.5\textwidth}

            \begin{terminalwindow}
!\shellcommand{g++ defines-param-example.cpp}!
!\shellcommand{./a.out}!
1, 3.14, text
            \end{terminalwindow}

        \end{column}

    \end{columns}

\end{frame}

\begin{frame}[fragile]

    \frametitle{Препроцессор: директива \\ \texttt{\#include}}

    \begin{columns}[T]

        \begin{column}{0.5\textwidth}

            \myinputlisting[minted language=cpp]
                {Presentations/02-Phases-of-translation/}
                {include-header.h}

            \myinputlisting[minted language=cpp]
                {Presentations/02-Phases-of-translation/}
                {include-source.cpp}

        \end{column}

        \begin{column}{0.5\textwidth}

            \begin{terminalwindow}
!\shellcommand{g++ -E include-source.cpp}!
# 0 "include-source.cpp"
# 0 "<built-in>"
# 0 "<command-line>"
# 1 "/usr/include/stdc-predef.h" 1 3 4
# 0 "<command-line>" 2
# 1 "include-source.cpp"
# 1 "include-header.h" 1
!\colorbox{yellow}{int a = 0;}!
# 2 "include-source.cpp" 2

int main() {
    int b = a;
    return 0;
}
            \end{terminalwindow}

        \end{column}

    \end{columns}

\end{frame}

\begin{frame}[fragile]

    \frametitle{Препроцессор: директива \\ \texttt{\#pragma}}

    \begin{columns}[T]

        \begin{column}{0.5\textwidth}

            \myinputlisting[minted language=cpp]
                {Presentations/02-Phases-of-translation/}
                {pragma-example.h}

            \myinputlisting[minted language=cpp]
                {Presentations/02-Phases-of-translation/}
                {pragma-example.cpp}

        \end{column}

        \begin{column}{0.5\textwidth}

            \begin{terminalwindow}
!\shellcommand{g++ -E pragma-example.cpp}!
# 0 "pragma-example.cpp"
# 0 "<built-in>"
# 0 "<command-line>"
# 1 "/usr/include/stdc-predef.h" 1 3 4
# 0 "<command-line>" 2
# 1 "pragma-example.cpp"
# 1 "pragma-example.h" 1


const auto PI = 3.14;
# 2 "pragma-example.cpp" 2




int main() {
    auto pi = PI;
}
            \end{terminalwindow}

        \end{column}

    \end{columns}

\end{frame}

\begin{frame}

    \frametitle{Компиляция}

    \centering

    \begin{tikzpicture}[node distance = 5cm]

        \node (source) {
            \includegraphics[width=1.7cm]{Presentations/images/source_code.png}
        };
        \node (binary) [right=of source] {
            \includegraphics[width=1.7cm]{Presentations/images/binary-file.png}
        };
        \node [below=0mm of binary] { \texttt{*.o} };

        \draw[->, ultra thick] (source.east) -- (binary.west)
            node[midway, above] {Компиляция};

    \end{tikzpicture}

\end{frame}

\begin{frame}[fragile]

    \frametitle{Компиляция}

    \begin{columns}[T]

        \begin{column}{0.25\textwidth}

            \myinputlisting[minted language=cpp]
                {Presentations/02-Phases-of-translation/}
                {comp-example.cpp}

        \end{column}

        \begin{column}{0.75\textwidth}

            \begin{terminalwindow}
!\shellcommand{g++ \colorbox{yellow}{-c} comp-example.cpp -o comp-example.o}!
!\shellcommand{objdump -f comp-example.o}!

comp-example.o:     file format !\colorbox{green}{elf64-x86-64}!
architecture: i386:x86-64, flags 0x00000011:
HAS_RELOC, HAS_SYMS
start address 0x0000000000000000
!\shellcommand{objdump \colorbox{yellow}{-tC} comp-example.o}!

comp-example.o:     file format elf64-x86-64

SYMBOL TABLE:
0000000000000000 l    df *ABS*  0000000000000000 comp-example.cpp
0000000000000000 l    d  .text  0000000000000000 .text
0000000000000000 l    d  .rodata        0000000000000000 .rodata
0000000000000000 g     F .text  000000000000000e !\colorbox{green}{PI()}!
000000000000000e g     F .text  000000000000001b !\colorbox{green}{main}!
            \end{terminalwindow}

        \end{column}

    \end{columns}

\end{frame}

\begin{frame}[fragile]

    \frametitle{Компиляция}

        \begin{terminalwindow}
!\shellcommand{objdump \colorbox{yellow}{-d} comp-example.o}!

comp-example.o:     file format elf64-x86-64


Disassembly of section .text:

0000000000000000 <_Z2PIv>:
   0:   55                      push   %rbp
   1:   48 89 e5                mov    %rsp,%rbp
   4:   f3 0f 10 05 00 00 00    movss  0x0(%rip),%xmm0        # c <_Z2PIv+0xc>
   b:   00
   c:   5d                      pop    %rbp
   d:   c3                      ret

000000000000000e <main>:
   e:   55                      push   %rbp
   f:   48 89 e5                mov    %rsp,%rbp
  12:   48 83 ec 10             sub    $0x10,%rsp
  16:   e8 00 00 00 00          call   1b <main+0xd>
  1b:   66 0f 7e c0             movd   %xmm0,%eax
  1f:   89 45 fc                mov    %eax,-0x4(%rbp)
  22:   b8 00 00 00 00          mov    $0x0,%eax
  27:   c9                      leave
  28:   c3                      ret
        \end{terminalwindow}

    \centering

\end{frame}

\begin{frame}[fragile]

    \frametitle{Линковка}

    \centering

    \begin{tikzpicture}[node distance=5cm]

        \matrix (objs) [row sep=0.5cm, column sep=0.1cm] {
            \node (bin1) {
                \includegraphics[width=1.5cm]{Presentations/images/binary-file.png}
            }; &
            \node (bin2) {
                \includegraphics[width=1.5cm]{Presentations/images/binary-file.png}
            }; \\
            \node (bin3) {
                \includegraphics[width=1.5cm]{Presentations/images/binary-file.png}
            }; &
            \node (bin4) {
                \includegraphics[width=1.5cm]{Presentations/images/binary-file.png}
            }; \\
        };
        \node [below=0mm of bin1] { \texttt{*.o} };
        \node [below=0mm of bin2] { \texttt{*.o} };
        \node [below=0mm of bin3] { \texttt{*.o} };
        \node [below=0mm of bin4] { \texttt{*.o} };

        \node (exe) [right=of objs] {
            \includegraphics[width=1.5cm]{Presentations/images/binary-file.png}
        };
        \node [below=0mm of exe] { \texttt{ELF} };

        \draw[->, ultra thick] (objs.east) -- (exe.west)
            node[midway, above] {Линковка};

    \end{tikzpicture}

\end{frame}

\begin{frame}[fragile]

    \frametitle{Линковка}

    \begin{columns}[T]

        \begin{column}{0.4\textwidth}

            \myinputlisting[minted language=cpp]
                {Presentations/02-Phases-of-translation/}
                {link-example.h}

            \myinputlisting[minted language=cpp]
                {Presentations/02-Phases-of-translation/}
                {link-example.cpp}

            \myinputlisting[minted language=cpp]
                {Presentations/02-Phases-of-translation/}
                {link-example-main.cpp}

        \end{column}

        \begin{column}{0.6\textwidth}

            \begin{terminalwindow}[||]
|\shellcommand{g++ -c link-example.cpp}|
|\shellcommand{g++ -c link-example-main.cpp}|
|\shellcommand{g++ link-example.o link-example-main.o}|
|\shellcommand{./a.out}|
Hello!
|\shellcommand{file a.out}|
a.out: ELF |\colorbox{green}{64-bit}| LSB executable, |\colorbox{green}{x86-64}|, version 1 (SYSV), dynamically linked, interpreter /lib64/ld-linux-x86-64.so.2, BuildID[sha1]=9e6a23c78b72c21d9e51a783409c9ec5064a8d55, for GNU/Linux 3.2.0, not stripped
            \end{terminalwindow}

        \end{column}

    \end{columns}

\end{frame}

\begin{frame}[fragile]

    \frametitle{Статическая библиотека}

    \centering

    \begin{tikzpicture}[node distance=5cm]

        \matrix (objs) [row sep=0.5cm, column sep=0.1cm] {
            \node (bin1) {
                \includegraphics[width=1.5cm]{Presentations/images/binary-file.png}
            }; &
            \node (bin2) {
                \includegraphics[width=1.5cm]{Presentations/images/binary-file.png}
            }; \\
            \node (bin3) {
                \includegraphics[width=1.5cm]{Presentations/images/binary-file.png}
            }; &
            \node (bin4) {
                \includegraphics[width=1.5cm]{Presentations/images/binary-file.png}
            }; \\
        };
        \node [below=0mm of bin1] { \texttt{*.o} };
        \node [below=0mm of bin2] { \texttt{*.o} };
        \node [below=0mm of bin3] { \texttt{*.o} };
        \node [below=0mm of bin4] { \texttt{*.o} };

        \node (exe) [right=of objs] {
            \includegraphics[width=1.5cm]{Presentations/images/binary-file.png}
        };
        \node [below=0mm of exe] { \texttt{lib*.a} };

        \draw[->, ultra thick] (objs.east) -- (exe.west)
            node[midway, above] {Создание архива};

    \end{tikzpicture}

\end{frame}

\begin{frame}[fragile]

    \frametitle{Статическая библиотека}

    \centering

    \begin{tikzpicture}[node distance=5cm]

        \matrix (objs) [row sep=0.5cm, column sep=0.1cm] {
            \node (bin1) {
                \includegraphics[width=1.5cm]{Presentations/images/binary-file.png}
            }; &
            \node (bin2) {
                \includegraphics[width=1.5cm]{Presentations/images/binary-file.png}
            }; \\
            \node (bin3) {
                \includegraphics[width=1.5cm]{Presentations/images/binary-file.png}
            }; &
            \node (lib) {
                \includegraphics[width=1.5cm]{Presentations/images/binary-file.png}
            }; \\
        };
        \node [below=0mm of bin1] { \texttt{*.o} };
        \node [below=0mm of bin2] { \texttt{*.o} };
        \node [below=0mm of bin3] { \texttt{*.o} };
        \node [below=0mm of lib]  { \texttt{lib*.a} };

        \node (exe) [right=of objs] {
            \includegraphics[width=1.5cm]{Presentations/images/binary-file.png}
        };
        \node [below=0mm of exe] { \texttt{ELF} };

        \draw[->, ultra thick] (objs.east) -- (exe.west)
            node[midway, above] {Линковка};

    \end{tikzpicture}

\end{frame}

\begin{frame}[fragile]

    \frametitle{Статичеcкая библиотека}

    \begin{columns}[T]

        \begin{column}{0.5\textwidth}

            \myinputlisting[minted language=cpp]
                {Presentations/02-Phases-of-translation/library/}
                {main.cpp}

            \myinputlisting[minted language=cpp]
                {Presentations/02-Phases-of-translation/library/}
                {foo.cpp}

            \myinputlisting[minted language=cpp]
                {Presentations/02-Phases-of-translation/library/}
                {bar.cpp}

        \end{column}

        \begin{column}{0.5\textwidth}

            \begin{terminalwindow}[||]
|\shellcommand{g++ -c bar.cpp -o bar.o}|
|\shellcommand{g++ -c foo.cpp -o foo.o}|
|\shellcommand{\colorbox{green}{ar -r libfoo.a} foo.o bar.o}|
|\shellcommand{g++ -c main.cpp -o main.o}|
|\shellcommand{g++ main.o \colorbox{green}{-L\$(pwd) -lfoo} -o main}|
|\shellcommand{./main}|
Hello, world!
            \end{terminalwindow}

        \end{column}

    \end{columns}

\end{frame}

\begin{frame}[fragile]

    \frametitle{Динамическая библиотека}

    \centering

    \begin{tikzpicture}[node distance=5cm]

        \matrix (objs) [row sep=0.5cm, column sep=0.1cm] {
            \node (bin1) {
                \includegraphics[width=1.5cm]{Presentations/images/binary-file.png}
            }; &
            \node (bin2) {
                \includegraphics[width=1.5cm]{Presentations/images/binary-file.png}
            }; \\
            \node (bin3) {
                \includegraphics[width=1.5cm]{Presentations/images/binary-file.png}
            }; &
            \node (bin4) {
                \includegraphics[width=1.5cm]{Presentations/images/binary-file.png}
            }; \\
        };
        \node [below=0mm of bin1] { \texttt{*.o} };
        \node [below=0mm of bin2] { \texttt{*.o} };
        \node [below=0mm of bin3] { \texttt{*.o} };
        \node [below=0mm of bin4] { \texttt{*.o} };

        \node (exe) [right=of objs] {
            \includegraphics[width=1.5cm]{Presentations/images/binary-file.png}
        };
        \node [below=0mm of exe] { \texttt{lib*.so} };

        \draw[->, ultra thick] (objs.east) -- (exe.west)
            node[midway, above, align=left] {Создание динамической \\ библиотеки};

    \end{tikzpicture}

\end{frame}

\begin{frame}[fragile]

    \frametitle{Динамическая библиотека}

    \centering

    \begin{tikzpicture}[node distance=5cm]

        \matrix (src) [row sep=0.5cm, column sep=0.1cm] {
            \node (bin1) {
                \includegraphics[width=1.5cm]{Presentations/images/binary-file.png}
            }; &
            \node (bin2) {
                \includegraphics[width=1.5cm]{Presentations/images/binary-file.png}
            }; \\
            \node (bin3) {
                \includegraphics[width=1.5cm]{Presentations/images/binary-file.png}
            }; &
            \node (lib0) {
                \includegraphics[width=1.5cm]{Presentations/images/binary-file.png}
            }; \\
        };
        \node [below=0mm of bin1] { \texttt{*.o} };
        \node [below=0mm of bin2] { \texttt{*.o} };
        \node [below=0mm of bin3] { \texttt{*.o} };
        \node [below=0mm of lib0] { \texttt{lib*.so} };

        \matrix (dst) [row sep=0.5cm, column sep=0.1cm, right=of src] {
            \node (lib1) {
                \includegraphics[width=1.5cm]{Presentations/images/binary-file.png}
            }; \\
            \node (exe) {
                \includegraphics[width=1.5cm]{Presentations/images/binary-file.png}
            }; \\
        };
        \node [below=0mm of exe] { \texttt{executable} };
        \node [below=0mm of lib1] { \texttt{lib*.so} };

        \draw[->, ultra thick] (src.east) -- (dst.west)
            node[midway, above, align=left] {Линковка};

    \end{tikzpicture}

\end{frame}

\begin{frame}[fragile]

    \frametitle{Динамическая библиотека}

    \begin{columns}[T]

        \begin{column}{0.4\textwidth}

            \myinputlisting[minted language=cpp]
                {Presentations/02-Phases-of-translation/library/}
                {main.cpp}

            \myinputlisting[minted language=cpp]
                {Presentations/02-Phases-of-translation/library/}
                {foo.cpp}

            \myinputlisting[minted language=cpp]
                {Presentations/02-Phases-of-translation/library/}
                {bar.cpp}

        \end{column}

        \begin{column}{0.6\textwidth}

            \begin{terminalwindow}[||]
|\shellcommand{g++ -c -fPIC bar.cpp -o bar.o}|
|\shellcommand{g++ -c -fPIC foo.cpp -o foo.o}|
|\shellcommand{g++ \colorbox{green}{-shared} bar.o foo.o -o libfoo.so}|
|\shellcommand{g++ -c main.cpp -o main.o}|
|\shellcommand{g++ main.o \colorbox{green}{-L\$(pwd) -lfoo -Wl,-rpath=\$(pwd)} -o main}|
|\shellcommand{./main}|
Hello, world!
            \end{terminalwindow}

        \end{column}

    \end{columns}

\end{frame}

\begin{frame}[fragile]

    \frametitle{Источники, документация \\ и литература}

    \begin{itemize}

        \item Статья про фазы трансляции С++ на cppreference:
            \url{https://en.cppreference.com/w/cpp/language/translation\_phases}

        \item Перевод статьи Beginner's guide to linkers на Хабре:
            \url{https://habr.com/ru/articles/150327/}

        \item Цикл статей про архитектуру компьютера с фокусом на выполнение
            процессов: \url{https://cpu.land/}

        \item Создание статических и динамических библиотек при помощи
            \texttt{gcc}: \url{https://renenyffenegger.ch/notes/development/languages/C-C-plus-plus/GCC/create-libraries/index}

    \end{itemize}

\end{frame}

\end{document}
