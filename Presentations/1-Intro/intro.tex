% !TeX TS-program = xelatex

\documentclass[compress]{beamer}

\usepackage{presentationtemplate}
\usepackage[fontsize=\footnotesize, askip=3mm]{terminal}
\usepackage[fontsize=\footnotesize, linenosfontsize=\tiny, askip=3mm]{mylisting}

\subtitle{Введение}

\begin{document}

\frame[plain]{\titlepage}

\begin{frame}

    \frametitle{Темы курса}

    \begin{itemize}

        \item Компиляция

            \begin{itemize}

                \item Фазы трансляции

                \item Системы сборки

                \item Кросс-компиляция

                \item Управление зависимостями

            \end{itemize}

        \item Система контроля версий (git)

        \item Типы объектов

            \begin{itemize}

                \item Фундаментальные типы

                \item Квалификаторы типов

                \item Операции и операторы

                \item Приведение типов

            \end{itemize}

        \item Условия и циклы

        \item Указатели и работа с памятью

    \end{itemize}

\end{frame}

\begin{frame}

    \frametitle{Темы курса}

    \begin{itemize}

        \item Функции

            \begin{itemize}

                \item Сигнатура функции

                \item Объявление и определение функции

                \item Рекурсия

                \item Перегрузка функций

            \end{itemize}

        \item Пространства имен

        \item Классы

            \begin{itemize}

                \item Поля и методы класса, конструкторы и деструкторы,
                    время жизни объектов, инициализация

                \item \texttt{const}, \texttt{mutable}, статические поля и методы

                \item Наследование

                \item Полиморфизм

            \end{itemize}

    \end{itemize}

\end{frame}

\begin{frame}

    \frametitle{Темы курса}

    \begin{itemize}

        \item Исключения

        \item Шаблоны

        \item Стандартная библиотека

            \begin{itemize}

                \item Строки, потоки ввода-вывода, работа с файлами

                \item Контейнеры

                \item Алгоритмы

                \item Умные указатели

            \end{itemize}

    \end{itemize}

\end{frame}

\begin{frame}[fragile]

    \frametitle{Hello, world!}

    \myinputlisting[minted language=cpp]
        {Presentations/1-Intro/}
        {hello-world.cpp}

    \begin{terminalwindow}[||]
|\shellcommand{g++ hello-world.cpp}|
|\shellcommand{./a.out}|
Hello, World!
    \end{terminalwindow}

\end{frame}

\begin{frame}[fragile]

    \frametitle{Стандарты языка C++}

    \begin{columns}

        \begin{column}{0.4\textwidth}

            \begin{itemize}

                \item C++98
                \item C++03
                \item C++11/C++0x
                \item C++14
                \item C++17
                \item C++20
                \item C++23
                \item C++26 (еще не выпущен)

            \end{itemize}

        \end{column}

        \begin{column}{0.6\textwidth}

            \begin{myinplacelisting}[minted language=cpp]
auto i = 3.14; // C++11

// C++14
auto glambda = [] (auto a) {
    return a;
};

// C++17
void f([[maybe_unused]] int a);

// C++20
bool c = a <=> b;

// C++23
std::format("i={}\n", i);
            \end{myinplacelisting}
        \end{column}

    \end{columns}

\end{frame}

\begin{frame}

    \frametitle{Список полезных источников \\ и ресурсов}

    \begin{enumerate}

        \item \url{https://en.cppreference.com/w/}

        \item \url{https://cplusplus.com/}

        \item Bjarne Stroustrup: The C++ Programming Language

        \item Scott Meyers: Effective Modern C++

        \item \url{https://isocpp.org/}

        \item \href{http://isocpp.github.io/CppCoreGuidelines/CppCoreGuidelines}{C++ Core Guidelines}

        \item Working Draft, Standard for Programming Language C++

        \item \href{https://www.youtube.com/@lectory\_fpmi}
            {Лекции преподавателя ФПМИ Ильи Мещерина}

    \end{enumerate}

\end{frame}

\end{document}
