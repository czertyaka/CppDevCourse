% !TeX TS-program = xelatex

\documentclass[compress]{beamer}

\beamertemplatenavigationsymbolsempty
\setbeamertemplate{headline}{}

% ==== Packages section ============================

\usepackage{xltxtra}
\usepackage[main=russian,english]{babel}

\setmainfont{DejaVu Serif}
\setsansfont{DejaVu Sans}
\setmonofont{Fira Mono}

\usepackage[export]{adjustbox}
\usepackage{tikz}

\usepackage[newfloat]{minted}
\usemintedstyle{borland}
\SetupFloatingEnvironment{listing}{name=Листинг}

\usepackage{fancyvrb}

% ==== Beamer customization section ================

\title{Введение в разработку на C++}
\subtitle{Введение}
\author{Пыхов~Олег~Александрович\inst{1} \\ \texttt{pyhovoleg31@gmail.com}}
\institute{
    \inst{1}%
    ООО~"Прософт-Системы"
}
\date{2024}
\titlegraphic{
    \includegraphics[scale=0.1,valign=t]{Presentations/images/university-logo.png}
    \includegraphics[scale=0.3,valign=t]{Presentations/images/prosoft-logo.png}
}
\logo{
    \begin{tikzpicture}[overlay,remember picture]
        \node[left=0.2cm] at (current page.32){
            \includegraphics[scale=0.15,valign=t]{Presentations/images/prosoft-logo.png}
            \includegraphics[scale=0.05,valign=t]{Presentations/images/university-logo.png}
        };
    \end{tikzpicture}
}

\setbeamertemplate{footline}[frame number]

\begin{document}

\frame[plain]{\titlepage}

\begin{frame}

    \frametitle{Темы курса}

    \begin{itemize}

        \item Компиляция

            \begin{itemize}

                \item Фазы трансляции

                \item Системы сборки

                \item Кросс-компиляция

                \item Управление зависимостями

            \end{itemize}

        \item Система контроля версий (git)

        \item Типы объектов

            \begin{itemize}

                \item Фундаментальные типы

                \item Квалификаторы типов

                \item Операции и операторы

                \item Приведение типов

            \end{itemize}

        \item Условия и циклы

        \item Указатели и работа с памятью

    \end{itemize}

\end{frame}

\begin{frame}

    \frametitle{Темы курса}

    \begin{itemize}

        \item Функции

            \begin{itemize}

                \item Сигнатура функции

                \item Объявление и определение функции

                \item Рекурсия

                \item Перегрузка функций

            \end{itemize}

        \item Пространства имен

        \item Классы

            \begin{itemize}

                \item Поля и методы класса, конструкторы и деструкторы,
                    время жизни объектов, инициализация

                \item \texttt{const}, \texttt{mutable}, статические поля и методы

                \item Наследование

                \item Полиморфизм

            \end{itemize}

    \end{itemize}

\end{frame}

\begin{frame}

    \frametitle{Темы курса}

    \begin{itemize}

        \item Исключения

        \item Шаблоны

        \item Стандартная библиотека

            \begin{itemize}

                \item Строки, потоки ввода-вывода, работа с файлами

                \item Контейнеры

                \item Алгоритмы

                \item Умные указатели

            \end{itemize}

    \end{itemize}

\end{frame}

\begin{frame}[fragile]

    \frametitle{Hello, world!}

    \begin{listing}[H]
        \inputminted[frame=single, fontsize=\footnotesize, linenos]{cpp}
            {Presentations/1-Intro/hello-world.cpp}
        \caption{hello-world.cpp}
    \end{listing}

    \begin{Verbatim}[frame=single, fontsize=\footnotesize, commandchars=\\\{\}]
$ \textbf{g++ hello-world.cpp}
$ \textbf{./a.out}
Hello, World!
    \end{Verbatim}

\end{frame}

\end{document}
