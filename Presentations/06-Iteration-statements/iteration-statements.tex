% !TeX TS-program = xelatex

\documentclass[compress, 8pt]{beamer}

\usepackage{presentationtemplate}
\usepackage[askip=3mm, bskip=3mm]{terminal}
\usepackage[linenosfontsize=\tiny, askip=3mm, bskip=3mm]{mylisting}
\usepackage{tcolorbox}

\newtcolorbox{task}{
    colback=yellow!50!white,
    boxrule=0.02cm,
    colframe=black,
    sharp corners,
    left=0mm,
    right=0mm,
    top=0mm,
    bottom=0mm,
    before upper={\textbf{Задание}:\:},
}

\title{Циклы}

\begin{document}

\frame[plain]{\titlepage}

\begin{frame}[fragile]

    \frametitle{Iteration statements}

    \hfill \break

    Циклы представляют собой последовательность инструкций, которая будет
    выполняться снова и снова до тех пор, пока не будет истинно специально
    заданное условие или пока не будет осуществлен принудительный выход из
    цикла.

    \begin{itemize}

        \item \textbf{for loop}
        \begin{myinplacelisting}[minted language=cpp]
for (/*объявление*/; /*условие*/; /*выражение*/)
{/*тело цикла*/};
        \end{myinplacelisting}

        \item \textbf{range-based for}
        \begin{myinplacelisting}[minted language=cpp]
for (/*объявление*/: /*выражение*/) {/*тело цикла*/};
        \end{myinplacelisting}

        \item \textbf{while loop}
        \begin{myinplacelisting}[minted language=cpp]
while (/*условие*/) {/*тело цикла*/};
        \end{myinplacelisting}

        \item \textbf{do-while loop}
        \begin{myinplacelisting}[minted language=cpp]
do {/*тело цикла*/} while(/*условие*/);
        \end{myinplacelisting}

    \end{itemize}

\end{frame}

\begin{frame}[fragile]

    \frametitle{Цикл \texttt{for}}

    \myinputlisting[minted language=cpp]
        {Presentations/06-Iteration-statements/for-loop-0/}
        {main.cpp}

    \begin{terminalwindow}
!\shellcommand{g++ main.cpp -o main}!
!\shellcommand{./main}!
Number of this iteration is 0
Number of this iteration is 1
Number of this iteration is 2
    \end{terminalwindow}

\end{frame}

\begin{frame}[fragile]

    \frametitle{Цикл \texttt{for}}

    Объявления и выражения в круглых скобках цикла for необязательны.
    Это может быть полезно, если необходимо объявить переменную счетчика
    вне цикла (так она будет доступна за его пределами):

    \begin{myinplacelisting}[minted language=cpp]
void foo(const int i);

int main() {
    int i = 0;
    for (; i < 3; i++) {
        std::cout << i << std::endl;
    }

    foo(i);
}
    \end{myinplacelisting}

\end{frame}

\begin{frame}[fragile]

    \frametitle{Цикл \texttt{for}}

    Можно менять переменную счетчика не в каждой итерации:

    \begin{myinplacelisting}[minted language=cpp]
bool foo();

int main() {
    for (int i = 10; i > 0;) {
        // пусть нам необходимо изменять
        // переменную i только в том случае
        // если функция foo вернула true
        if (foo()) {
            i--;
        }
    }
}
    \end{myinplacelisting}

\end{frame}

\begin{frame}[fragile]

    \frametitle{Цикл \texttt{for}}

    С помощью цикла \verb|for| можно составить бесконечный
    цикл\footnotemark{}:

    \footnotetext{Несмотря на то, что бесконечные циклы используются в
    программировании на C++ достаточно часто (например, когда завершение
    работы программы не предусматривается), стандарт языка рассматривает
    их как неопределенное поведение.}

    \begin{myinplacelisting}[minted language=cpp]
void foo();

int main() {
    for (;;) {
        // этот цикл не закончится никогда
        foo();
    }
}
    \end{myinplacelisting}

\end{frame}

\begin{frame}[fragile]

    \frametitle{Цикл \texttt{for}}

    \begin{task}
        Попробуйте найти логическую ошибку в коде на листинге,
        которая приведет к неопределенному поведению, и предложите
        исправление
    \end{task}

    \begin{myinplacelisting}[minted language=cpp]
bool foo();

int main() {
    for (unsigned i = 0; i == 10; i++) {
        if (foo()) {
            i += 2;
        }
    }
}
    \end{myinplacelisting}

\end{frame}

\begin{frame}[fragile]

    \frametitle{Цикл \texttt{for}}

    \begin{task}
        Попробуйте найти логическую ошибку в коде на листинге,
        которая приведет к неопределенному поведению, и предложите
        исправление.
        Назовите оба вида неопределенного поведения, которые возникнут
        при выполнении программы.
    \end{task}

    \begin{myinplacelisting}[minted language=cpp]
#include <iostream>

int main() {
    signed char i = 0;
    for (; i < 200; i++) {
        std::cout << "i = " << i << std::endl;
    }
}
    \end{myinplacelisting}

\end{frame}

\begin{frame}[fragile]

    \frametitle{Цикл \texttt{while}}

    \hfill \break

    В случае, когда для реализации цикла переменная кажется излишней, можно
    использовать цикл \verb|while|.
    Тело цикла \verb|while| будет выполняться до тех пор, пока истинно
    условие в круглых скобках после \verb|while|.

    \begin{myinplacelisting}[minted language=cpp]
#include <iostream>

bool do_continue();
int get_x();

int main() {
    while(do_continue()) {
        const int x = get_x();
        std::cout << "x^2 = " << (x * x) << std::endl;
    }

    // эквивалентно

    for (; do_continue();) {
        const int x = get_x();
        // ...
    }
}
    \end{myinplacelisting}

\end{frame}

\begin{frame}[fragile]

    \frametitle{Цикл \texttt{do-while}}

    В цикле \verb|while| условие проверяется до того, как тело цикла
    выполнится в первый раз.
    Если необходимо, чтобы первая итерация выполнялась безусловно, а
    все последюущие — по условию, то можно использовать цикл
    \verb|do-while|.

    \begin{myinplacelisting}[minted language=cpp]
#include <iostream>

bool do_continue();
int get_x();

int main() {
    do {
        const int x = get_x();
        std::cout << "x^2 = " << (x * x) << std::endl;
    } while(do_continue());
}
    \end{myinplacelisting}

\end{frame}

\begin{frame}[fragile]

    \frametitle{Выход из цикла}

    \hfill \break

    В теле цикла ключевое слово \verb|break| используется как оператор
    принудительного выхода из цикла.
    Такая необходимость может появиться, если невозможно или
    неудобно учесть все сценарии прекращения итераций в блоке условий
    (круглых скобках \verb|for| и \verb|while|).
    Как только выполнение дойдет до ключевого слова \verb|break|,
    управление будет передано коду, написанному ниже цикла.

    \begin{myinplacelisting}[minted language=cpp]
#include <iostream>

bool do_continue();
int get_x();

int main() {
    do {
        const int x = get_x();
        std::cout << "x^2 = " << (x * x) << std::endl;
        if (!do_continue()) {
            break;
        }
    } while(true);
}
    \end{myinplacelisting}
\end{frame}

\begin{frame}[fragile]

    \frametitle{Пропуск итерации}

    \hfill \break

    Ключевое слово \verb|continue| прерывает текущую итерацию,
    после чего продолжается выполнение следующих итераций:

    \begin{task}
        Объясните, зачем потребовался пропуск итерации
    \end{task}

    \begin{myinplacelisting}[%
        minted language=cpp,
        minted options={
            fontsize=\normalsize,
            breaklines,
            breakanywhere,
            linenos,
            escapeinside=??,
            numbersep=2pt
        },
    ]
#include <iostream>
#include <cstdint>

#define MAX_X 11

bool do_continue();
std::int8_t get_x();

int main() {
    do {
        const std::int8_t x = get_x();
        if (x > MAX_X || x < -MAX_X) {
            continue;
        }

        std::cout << "x^2 = " << (x * x) << std::endl;
    } while(do_continue());
}
    \end{myinplacelisting}

\end{frame}

\end{document}
